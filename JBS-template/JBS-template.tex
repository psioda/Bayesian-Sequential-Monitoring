\documentclass[12pt]{article}
\usepackage{amsmath}
\usepackage{graphicx,psfrag,epsf}
\usepackage{enumerate}
\usepackage{natbib}
\usepackage{url} 
\usepackage{color, colortbl}
\usepackage{booktabs}
\usepackage{geometry}
\geometry{margin=1in}
\usepackage{setspace}
\doublespacing

\usepackage{amsmath,amsfonts}
\usepackage{mathrsfs,color,nicefrac,multirow,colortbl}

\newcommand{\etal}{\textit{et al.}}
\newcommand{\inc}[2]{ \ifthenelse{\equal{#1}{1}}{\input{./sections/#2}}{ } }
\definecolor{Gray}{gray}{0.9}

\begin{document}


	\begin{center}
		\textbf{A Structured Framework for Adaptively Incorporating External\\ 
		         Evidence in Sequentially Monitored Clinical Trials} \\

		\vspace{1cm}
			Evan Kwiatkowski\textsuperscript{1}, Eugenio Andraca-Carrera\textsuperscript{2}, 
			Mat Soukup\textsuperscript{2}, Matthew A. Psioda\textsuperscript{1}\\

		\vspace{1cm}
			\textsuperscript{1} Department of Biostatistics, University of \\
			North Carolina, McGavran-Greenberg Hall, CB\#7420 \\

		\vspace{1cm}
			\textsuperscript{2}{Division of Biometrics VII, Office of Biostatistics, Center for Drug Evaluation and Research, US Food and Drug Administration, Silver Spring, Maryland, USA}
	\end{center}


\vspace{1.5cm}
\noindent
\textbf{Abstract}:
{We present a Bayesian framework for sequential monitoring that allows for use of external data, and that 
can be applied in a wide range of clinical trial applications. The basis for this framework is the 
idea that, in many cases, specification of priors used for sequential monitoring and the stopping criteria can be semi-algorithmic byproducts of the trial hypotheses and relevant external data, simplifying the process of prior elicitation. Monitoring priors are defined using the family of generalized normal distributions which comprise a flexible class of priors, naturally allowing one to construct a prior that is peaked or flat about the parameter values thought to be most likely. External data are incorporated into the monitoring process though mixing an a priori skeptical prior with an enthusiastic prior using a weight that can be fixed or adaptively estimated. In particular, we introduce an \textcolor{black}{adaptive monitoring prior} for efficacy evaluation which dynamically weighs skeptical and enthusiastic prior components based on the degree to which observed data \textcolor{black}{are} consistent with an enthusiastic perspective. The proposed approach allows for prospective and pre-specified use of external data in the monitoring procedure.
%
We illustrate the method for both single-arm and two-arm randomized controlled trials. For the latter case, we \textcolor{black}{also include a retrospective analysis of actual trial data using the proposed adaptive sequential monitoring procedure}. Both examples are motivated by completed pediatric trials, and the designs incorporate information from adult trials to varying degrees. Preposterior analysis \textcolor{black}{and frequentist operating characteristics} of each trial design \textcolor{black}{are discussed.}}


\vspace{1.5cm}
\noindent
\textbf{Keywords}:
 Adaptive Trial Design, Bayesian Sequential Monitoring, Information Borrowing, Pediatric Trials, Skeptical Prior.


\newpage

\renewcommand\thesection{\arabic{section}}
\renewcommand\thesubsection{\arabic{section}.\arabic{subsection}}


%%%%%%%%%%%%%%%%%%%%%%%%%%%%%%%%%%%%%%%%%%%%%%%%%%%%%%%%%%%%%%%%%%%%%%%%%%%%%%%%%%%%%%%%%%%%%%%%%%%%%%%%%%%%%%%%%%%
%%%%%%%%%%%%%%%%%%%%%%%%%%%%%%%%%%%%%%%%%%%%%%%%%%%%%%%%%%%%%%%%%%%%%%%%%%%%%%%%%%%%%%%%%%%%%%%%%%%%%%%%%%%%%%%%%%%
%%%%%%%%%%%%%%%%%%%%%%%%%%%%%%%%%%%%%%%%%%%%%%%%%%%%%%%%%%%%%%%%%%%%%%%%%%%%%%%%%%%%%%%%%%%%%%%%%%%%%%%%%%%%%%%%%%%
%%%%%%%%%%%%%%%%%%%%%%%%%%%%%%%%%%%%%%%%%%%%%%%%%%%%%%%%%%%%%%%%%%%%%%%%%%%%%%%%%%%%%%%%%%%%%%%%%%%%%%%%%%%%%%%%%%%
%%%%%%%%%%%%%%%%%%%%%%%%%%%%%%%%%%%%%%%%%%%%%%%%%%%%%%%%%%%%%%%%%%%%%%%%%%%%%%%%%%%%%%%%%%%%%%%%%%%%%%%%%%%%%%%%%%%
%%%%%%%%%%%%%%%%%%%%%%%%%%%%%%%%%%%%%%%%%%%%%%%%%%%%%%%%%%%%%%%%%%%%%%%%%%%%%%%%%%%%%%%%%%%%%%%%%%%%%%%%%%%%%%%%%%%

%\section{Introduction} \label{s:introduction} 
%
%This is the introduction....

%%%%%%%%%%%%%%%%%%%%%%%%%%%%%%%%%%%%%%%%%%%%%%%%%%%%%%%%%%%%%%%%%%%%%%%%%%%%%%%%%%%%%%%%%%%%%%%%%%%%%%%%%%%%%%%%%%%
%%%%%%%%%%%%%%%%%%%%%%%%%%%%%%%%%%%%%%%%%%%%%%%%%%%%%%%%%%%%%%%%%%%%%%%%%%%%%%%%%%%%%%%%%%%%%%%%%%%%%%%%%%%%%%%%%%%
%%%%%%%%%%%%%%%%%%%%%%%%%%%%%%%%%%%%%%%%%%%%%%%%%%%%%%%%%%%%%%%%%%%%%%%%%%%%%%%%%%%%%%%%%%%%%%%%%%%%%%%%%%%%%%%%%%%
%%%%%%%%%%%%%%%%%%%%%%%%%%%%%%%%%%%%%%%%%%%%%%%%%%%%%%%%%%%%%%%%%%%%%%%%%%%%%%%%%%%%%%%%%%%%%%%%%%%%%%%%%%%%%%%%%%%
%%%%%%%%%%%%%%%%%%%%%%%%%%%%%%%%%%%%%%%%%%%%%%%%%%%%%%%%%%%%%%%%%%%%%%%%%%%%%%%%%%%%%%%%%%%%%%%%%%%%%%%%%%%%%%%%%%%
%%%%%%%%%%%%%%%%%%%%%%%%%%%%%%%%%%%%%%%%%%%%%%%%%%%%%%%%%%%%%%%%%%%%%%%%%%%%%%%%%%%%%%%%%%%%%%%%%%%%%%%%%%%%%%%%%%%
%\section{Methods} \label{s:methods}    
%\subsection{Preliminaries} \label{s:prelim}
%\subsubsection{Bayesian Hypothesis Testing} \label{s:bht}
%
%Please see Table~\ref{tab:pluto}. Also, Please see Figure~\ref{HEATMAP}.





\section{Introduction}

In the United States, sponsors of trials evaluating new drugs, biologics, and devices are required to monitor these trials \citep{FDA2006}. 
%
While monitoring of trials takes on various forms, one form of monitoring is to assess the safety and efficacy of a product in an ongoing trial at pre-defined intervals (i.e. interim analyses), typically through an independent data monitoring committee. 
%
The most commonly used statistical approach for interim analyses uses frequentist group sequential methods in which Type I error for testing a null and alternative hypothesis is distributed across the set of interim analyses to ensure overall Type I error control for establishing the efficacy or futility of a product \citep{Jennison2000}.  
%
%Bayesian monitoring of sequential trials is an alternative to the frequentist approach. 
Alternatively, in Bayesian sequential monitoring, data can be monitored on a continual basis and the evidence in favor (or against) a hypothesis can be evaluated against a single standard without penalty \citep{Spiegelhalter1993}. %
%Under this paradigm, data is monitored and collection stopped when any of the following conditions are met (a) a sufficiently skeptical person is convinced the alternative hypothesis is true (b) a sufficiently enthusiastic person is convinced the alternative is false or the benefit of treatment is not as high as expected (c) the probability of eventually proving the alternative is true is sufficiently low or (d) all resources allocated have been exhausted \citep{Spiegelhalter1993} .

In the clinical development of a therapeutic product, external forces may impede a clinical trial from reaching its objective (e.g. difficulty in enrolling, limited patient populations, long latency in observing the outcome of interest). 
%
This challenge is especially apparent in cases where the disease for which the investigational product (IP) is an intended treatment is rare or where the focus is on \textcolor{black}{a pediatric population}. 
%
In settings where patients are difficult to enroll, and therefore meaningful numbers of patients will complete follow-up prior to the trial reaching full enrollment, the concept of frequently monitoring interim data to determine whether a trial (or enrollment) can be stopped becomes appealing. 
%
As such, the use of Bayesian sequential methods, rooted in the likelihood principle and thus completely consistent with frequent or even continual data monitoring, provide an ideal basis from \textcolor{black}{which to develop novel trial designs - a goal under the 21st Century CURES Act \citep{USCongress2016}.}
% \textcolor{red}{...which to develop Complex Innovative Designs (CIDs) – a performance goal set forth under PDUFA VI legislation\citep{FDA2017}}.
%

\textcolor{red}{In settings where pertinent preexisting data are available, Bayesian methods provide a natural approach for incorporating that information, via a prior distribution, into the design and analysis of a future trial.
%
In traditional frequentist approaches to design, preexisting data can provide external information to inform a plausible, clinically meaningful 
choice for the value of the treatment effect for power calculations for a new study.
%
For the Bayesian approach, external information is typically translated into a prior distribution that characterizes what is currently thought about the treatment effect and the strength of evidence that the effect meets certain criteria (e.g., the treatment is helpful regarding the outcome of interest).
%
%This is comparable to the value used in frequentist analyses to inform power calculations.
%
%External evidence refers to the use of external information in probabilistic statements regarding the treatment effect (e.g. determination of treatment efficacy). 
%
%In Bayesian methods, external evidence can be incorporated via a prior distribution which is informed by external information and is used in the analysis of the treatment effect.
}

%
%\textcolor{red}{The potential use of Bayesian methods is discussed throughout the draft guidance \citep{FDA_CID}, created in response to the 21st Century CURES Act \citep{USCongress2016}, including the use of Bayesian methods to extrapolate information from adult patients to pediatric settings as one type of possible CID (see Section III, Part C)}.

%Enrolling the number of patients needed to adequately address objectives of a clinical trial can be difficult.
%% 
%This challenge is especially apparent in cases where the disease for which the investigational product (IP) is an 
%intended treatment is rare or where the focus is on pediatric disease.
%%
%There is increased interest in innovative ways to perform trials in these settings to facilitate obtaining substantial
%evidence of treatment benefit as efficiently as possible given the challenge of enrolling patients in a timely fashion.
%%
%Indeed, this is exemplified by the draft guidance from the United States Food and Drug Administration (FDA) which outlines
%procedures for interacting with FDA on complex innovative trial designs \citep{FDA_CID}, abbreviated CIDs, and the FDA CID Pilot 
%Program which was initiated to fulfill a performance goal set forth under the PDUFA IV legislation.
%%
%In the draft guidance (see Section II), CIDs are defined as ``trial designs that have rarely or never been used 
%to date to provide substantial evidence of effectiveness in new drug applications or biologics license applications.''
%%
%The potential use of Bayesian methods are discussed throughout the draft guidance and, in particular, the use of Bayesian methods
%to extrapolate information from adult patients to pediatric settings is given as one type of possible CID (see Section 
%II, Part C).
%
%In settings where patients are difficult to enroll, and therefore meaningful numbers of patients will complete follow up prior to 
%the trial reaching full enrollment, the concept of frequently monitoring interim data to determine whether a trial 
%(or enrollment) can be stopped becomes appealing. 
%%
%A promising interim analysis that provides substantial evidence of efficacy may justify ending enrollment, while enrolled patients may continue 
%to receive the treatment for the pre-planned period of exposure. 
%%
%A discouraging interim analysis that provides substantial evidence of futility may justify ending enrollment, and may call for enrolled patients 
%who are ongoing in the trial to be transitioned off the IP (i.e., termination of investigation of the treatment). 
%%
%For the Bayesian, the question becomes ``From what prior perspective must the evidence be substantial to justify one of the two actions described above?'' 
%%
%This motivates the notion of skeptical and enthusiastic perspectives.
%%
%The use of monitoring based on changing the opinion of skeptical and enthusiastic observers has been described as overcoming a 
%handicap \citep{Freedman1989} and providing a brake on the premature termination of trials \citep{Fayers1997}, and as 
%constructing ``an adversary who will need to be disillusioned by the data to stop further experimentation" \citep{Spiegelhalter1994}. 
%
%
%In the author's opinion, the use of Bayesian sequential methods, rooted in the likelihood principle and thus completely consistent with 
%frequent or even continual data monitoring, provide an ideal basis from which to develop CIDs 
%for these settings.
%%
%Moreover, in settings where pertinent preexisting data are available, Bayesian methods inarguably provide a natural 
%approach for incorporating that information via a prior distribution into the design and analysis of a future trial. 

In this paper, we propose a strategy for designing sequentially monitored clinical trials that entails eliciting 
priors used to monitor enrollment and/or data collection (i.e., monitoring priors) and stopping criteria that can 
be derived in a semi-automatic fashion based on standard inputs that are required for trial planning. 
%
These inputs include (1) the boundary null value for the treatment effect, (2) a plausible, clinically meaningful 
value for the treatment effect, and (3) a criteria for what constitutes a compelling demonstration of efficacy. 
%
In principle, the plausible, clinically meaningful value for the treatment effect should be informed by relevant external data. 


A key contribution of this work is \textcolor{black}{the provision of} structured definitions of skeptical and enthusiastic perspectives that can be used 
to inform early stopping decisions in favor of efficacy and futility, respectively. Skeptical and enthusiastic priors are developed using the generalized normal family of distributions. 
	%
This flexible family includes the normal distribution as a special case, and provides the capacity to construct monitoring 
priors that reflect nuanced prior opinion about the treatment effect. 
	%
A conditional-marginal prior factorization is proposed for settings where there are one or more nuisance 
parameters, and we illustrate how prior information can be used in both marginal 
distribution for the treatment effect and conditional distribution for the nuisance parameters. 
%
\textcolor{black}{The structured definitions of skeptical and enthusiastic perspectives form the basis for an adaptive monitoring prior to be used for efficacy evaluations monitoring when there is a desire to incorporate prior information into the monitoring process. The prospective use of external data in a pre-specified design provides novelty beyond conducting sensitivity analyses with different priors, and provides a pathway for innovative designs.}

We perform simulation-based preposterior analysis to examine a variety of operating characteristics for the proposed design framework,
and to understand how key operating characteristics are influenced by the frequency of monitoring.
%	%
%\textcolor{red}{General feedback: Compare this text against the previous version. There was a good deal of 
%confusing/inaccurate language -- e.g.,  coverage probability for the treatment effect or probability of early stopping
%at an interim or final analysis. We need all sentences to be 100\% conceptually and mathematically precise.}
	%
Specifically, we estimate the probability of stopping early at an interim analysis due to a compelling demonstration of efficacy or futility, \textcolor{black}{and} the sample size \textcolor{black}{at the interim and final analyses.}
	%
In most cases, patients will be ongoing in the trial at the time interim data are obtained that lead to 
ending enrollment as a result of a compelling demonstration of efficacy. 
	%
It is our assumption that in most cases these patients will complete the study protocol and, accordingly, we also explore 
the degree that interim evidence changes on average, once final data are available.	

Bayesian sequential designs are often restricted to have explicit frequentist properties \citep{Ventz2015, Zhu2015}. 
	%
Prior work has shown such restrictions can result in Bayesian and frequentist designs that have stopping rules which are 
nearly identical \citep{Stallard2020, Kopp-Schneider2020, Zhu2019}.
	%
While it is possible to calibrate a Bayesian design to have specific frequentist operating characteristics, we do not advocate
for that strategy.
	%
Instead, we propose a Bayesian framework that leverages what the authors argue is an intuitive criteria for stopping enrollment 
and/or data collection at any point (based on posterior inference using a consistent criteria for a compelling demonstration) without 
explicit focus on strict type I error control -- something that is not achievable when prior information is incorporated into 
the analysis \citep{Psioda2018}.
	%
%\textcolor{green}{Please check the accuracy of the following claim.}
%As will be illustrated, when prior information is not incorporated into the analysis, the designs operating characteristics 
%are ``well-calibrated" as discussed by \cite{Grieve2016}. 

%\textcolor{green}{Please give a more granular breakdown of what is in the paper. In particular, please describe the content 
%in Section \ref{sec:methods} in more depth. You can go into what is in the subsections.}
This paper is organized as follows: %
Section \ref{sec:preliminaries} reviews Bayesian hypothesis testing using posterior probabilities and the use of  skeptical and enthusiastic priors for efficacy and futility monitoring. 
%
Section \ref{sec:mps} presents a method for parameterizing monitoring priors the generalized normal distribution and for incorporating prior information into the monitoring priors, and a method to specify priors for nuisance parameters.
%
Examples are given in Section \ref{sec:examples}, with Section \ref{sec:example1} presenting an example based on a single-arm trial and Section \ref{sec:example2} presenting an example based on a two-arm randomized, controlled trial. \textcolor{black}{The purpose of Section \ref{sec:example1} is to demonstrate using skeptical and enthusiastic priors that are peaked or flat about the parameter values thought to be most likely, and the purpose of Section \ref{sec:example2} is to demonstrate the adaptive monitoring prior in a more complicated setting that involves specifying a prior for a nuisance parameter.} \textcolor{red}{Section \ref{sec:non-informative} contains a comparison of the adaptive monitoring prior with a standard design using a non-informative prior.}
%Conclusions from Bayesian inferences are not affected by the frequency with which the data are monitored, and such interim analyses should be encouraged to terminate data collection if appropriate \citep{Berry1989,Berry1993,Spiegelhalter1994}. The Bayesian perspective is based on the likelihood principle which asserts that any data with the same likelihood function should lead to the same conclusion \citep{Berger1988}. In the context of sequential analysis of clinical trials, the stopping rule that leads to the termination of data collection is irrelevant to the conclusion \citep{Barnard1947,Anscombe1963,Cornfield1966a,Cornfield1966b}. 
%In contrast ... add stuff Frequentist group sequential designs rely on pre-specified interim analyses and spending functions to maintain desired trial operating characteristics \citep{Pocock1977,OBrien1979}.
%Interpretation of conclusions from the Bayesian perspective is natural when specification of prior distributions are intuitively related to the research objectives. Skeptical and enthusiastic monitoring priors are used to evaluate efficacy and futility at an interim analysis \citep{Freedman1989,Freedman1992,Spiegelhalter1993,Fayers1997}. The most common Bayesian metrics for assessing evidence at interim analyses are posterior probabilities and predictive probabilities, and the choice of metric depends on the research objective. Posterior probabilities assess the level of evidence in favor of the null or alternative hypothesis, and predictive probabilities determine the capacity for the trial to show convincing evidence in favor of the alternative hypothesis if more outcomes are ascertained. 
\section{Methods}\label{sec:methods}

\subsection{Preliminaries}\label{sec:preliminaries}
\subsubsection{Bayesian Hypothesis Testing}
Consider a clinical trial application where the primary objective is to test a hypothesis about an unknown quantity of interest which we denote by $\theta$, with possible values
for $\theta$ falling in the parameter space $\Theta$.
%
For example, in a single-arm trial with a binary response endpoint, $\theta \in (0,1)$ may be the response probability associated with patients receiving the IP.
%
In a two-arm trial with a binary response endpoint, $\theta \in (-1,1)$ may be the difference in response probabilities between patients receiving the IP and those receiving the control treatment (e.g., placebo).

Throughout the paper we will let $\mathbf{D}$ represent the data collected in a trial at some point in time. 
%
For example, for the two-arm trial example above and assuming no covariates other than the treatment indicator, $\mathbf{D}=\left\{y_i,z_i:i=1,...,n\right\}$ where $y_i$ is an indicator of response for patient $i$ and $z_i$ is an indicator for whether patient $i$ was assigned the IP.
%
We use the generic representation $p(\mathbf{D}|\theta,\eta)$ to reflect the density or mass function for the collective data $\mathbf{D}$ as 
a function of $\theta$ and potential nuisance parameters $\eta$, which could be multi-dimensional.
%
For the two-arm trial example, $\eta$ might correspond to the response probability for patients receiving the control treatment or some transformation thereof. 
	%
For ease of exposition, for the remainder of Section~\ref{sec:preliminaries} we will focus on the case where $\theta$ is the only unknown parameter.

Consider the hypothesis $H_0:\theta\in\Theta_{0}$ versus $H_1:\theta\in\Theta_{1}$. The posterior probability that $\theta\in\Theta_i$ is given by
\begin{align}\label{eq:postprob}
P(\theta\in\Theta_i|\mathbf{D})
=\frac{\int_{\Theta_i} p(\mathbf{D}|\theta)\pi (\theta)d\theta}{\int_{\Theta}p(\mathbf{D}|\theta)\pi(\theta) d\theta}
\end{align}
where $p(\mathbf{D}|\theta)$ is commonly referred to as the likelihood for $\theta$ and $\pi(\theta)$ is its prior distribution. 
%
% To make clear that the likelihood is a function of $\theta$, we denote it as $\mathcal{L}(\theta)$.
%
%\textcolor{red}{All of the appendices need to be moved to Supplemental Online Appendices in the format of the Biometrics paper I sent. If not, they are 
%considered an appendix to the paper and also considered towards the length maximum. Hence, we have to move them to a separate PDF file. }
%
We will also refer to $P(\theta\in\Theta_i|\mathbf{D})$ as the posterior probability of hypothesis $H_i$.
%
See Appendix A for a brief discussion of the appropriateness of referring to $P(\theta\in\Theta_i|\mathbf{D})$
as the posterior probability of hypothesis $H_i$.

%see \eqref{eq:equation1} and \eqref{eq:equation2} in Appendix Section \ref{sec:hypothesis}).

\subsubsection{Formalizing the Statistical Concept of a Compelling Demonstration}\label{sec:sub_evid}
Consider the one-sided hypotheses $H_0: \theta \le \theta_0$ versus $H_1: \theta > \theta_0$ for fixed $\theta_0$, which we refer to as the boundary null value.
	%
Often in Bayesian hypothesis testing, one rejects the null hypothesis when $P(\theta > \theta_0 |\mathbf{D})$ exceeds 
a prespecified threshold.
%
Let $\epsilon$ represent \textit{insignificant residual probabilistic uncertainty} regarding a claim. 
 %
 Define $1-\epsilon$ to be the threshold for posterior probabilities in favor of the claim 
(e.g., that $\theta > \theta_0$), such that posterior probabilities above $1-\epsilon$ are considered as providing \textit{a compelling demonstration} 
that the claim is true. 
%
Leveraging common practice, we will use $\epsilon=0.025$ for the examples presented herein 
so that $1-\epsilon=0.975$ is the threshold that determines when evidence of a claim is compelling.
%
Our purpose in this paper is not to debate the appropriateness of using $0.975$ as a threshold for defining a compelling demonstration, but rather to develop a strategy for prior elicitation that leverages an accepted threshold to simplify prior elicitation for sequentially monitored trials in hopes that this may facilitate the use of sequential monitoring more broadly and consistently.

%%\begin{mydef}
Formally, we say that an individual whose belief is summarized by the distribution $\pi\left(\theta\right)$ is \textit{all but convinced} that $H_i$ is true if 
\textcolor{red}{
\begin{equation}\label{eq:compellingevidence}
		P_\pi(\theta\in\Theta_i) = 1-\epsilon,
\end{equation} 
}
where the subscript $\pi$ in \eqref{eq:compellingevidence} is simply to indicate that the probability is calculated based on $\pi\left(\theta\right)$ which could be either a prior
or posterior distribution.
%%\end{mydef}

\subsubsection{Skeptical and Enthusiastic Monitoring Priors}\label{sec:MP}
Monitoring priors are used for interim analyses of the data, and the purpose of monitoring priors is to help answer the question ``Is the evidence compelling enough to stop enrollment for the trial, or possibly end it altogether?''
%
A promising interim analysis that provides a compelling demonstration of efficacy may justify ending enrollment, while enrolled patients may continue to receive the treatment for the pre-planned period of exposure. 
%
A discouraging interim analysis that provides a compelling demonstration of futility may justify ending enrollment, and may call for enrolled patients who are ongoing in the trial to be transitioned off the IP (i.e., termination of investigation of the treatment). 
%
For the Bayesian, the question becomes ``From what prior perspective must the evidence be compelling to justify one of the two actions described above?'' 
%
This motivates skeptical and enthusiastic monitoring priors, which represent two extreme but plausible beliefs about the quantity of interest $\theta$ relative to the hypotheses considered.
%
%The use of monitoring based on changing the opinion of skeptical and enthusiastic observers has been described as ``overcoming a handicap" \citep{Freedman1989} and ``providing a brake on the premature termination of trials" \citep{Fayers1997}, and as constructing ``an adversary who will need to be disillusioned by the data to stop further experimentation" \citep{Spiegelhalter1994}. 


Having formalized concepts for \textit{a compelling demonstration} and  being \textit{all but convinced} of a claim, we now can develop a structured framework for constructing skeptical and enthusiastic monitoring priors which will be used to determine early stopping rules for efficacy and futility, respectively.
%
Consider again the hypotheses $H_0: \theta \le \theta_0$ versus $H_0: \theta > \theta_0$ where $\theta_0$ represents a 
treatment effect of interest and let $\theta_1>\theta_0$ represent a plausible, clinically meaningful effect.
%
Define an enthusiastic prior, denoted as $\pi_{E}(\theta)$, as a prior consistent with $\theta_1$ being the most 
likely value of $\theta$ (i.e., the prior mode) and that reflects the belief of an observer who is 
\textit{all but convinced} that $H_1$ is true a priori. 
%
Formally, this is defined as satisfying (i) $\text{argmax}_\theta~\pi_E(\theta)=\theta_1$
and (ii) $P_E(\theta >\theta_0)=1-\epsilon$, where the subscript $E$ indicates that the probability is 
based on $\pi_{E}(\theta)$.
%
Similarly, define a skeptical prior, denoted as $\pi_{S}(\theta)$, as a prior consistent with $\theta_0$ being the most 
likely value of $\theta$ and that reflects the belief of an observer who is \textit{all but convinced} that 
$\theta <\theta_1$ is true a priori. 
%
Formally, this is defined as the prior $\pi_{S}(\theta)$ satisfying
(iii) $\text{argmax}_\theta~\pi_S(\theta)=\theta_0$  and (iv) $P_S(\theta <\theta_1)=1-\epsilon$.
%
In what follows we refer to (i) and (iii) as \textit{mode value constraints} and (ii) and (iv) as \textit{tail-probability constraints}, respectively.

Note that the proposed development of the skeptical prior does not generally reflect skepticism regarding whether the alternative hypothesis is true. 
%
Indeed, assuming a symmetric skeptical prior is elicited (as we propose), the \textit{induced} prior probabilities on the hypotheses
satisfy $p(H_0) =  p(H_1)$. (See Appendix A.)
%
%See Appendix XX of the Supplemental Materials for more exposition on this point.
%
%\textcolor{red}{Please make sure the first appendix touches on this point.}
%
Thus, the skeptical prior simply reflects skepticism regarding the possibility of large treatment effects but is
otherwise consistent with clinical equipoise regarding the two hypotheses.
%

The totality of evidence in favor of a hypothesis is influenced by 
the prior distribution used for analysis.
%
It is natural that one would stop a trial early in favor of efficacy or futility when the evidence in favor of the appropriate claim is compelling
to a sufficiently skeptical or enthusiastic observer, respectively, as defined above.
%
For example, if at any point data sufficiently convince an observer whose prior belief is in accordance with $\pi_{S}(\theta)$ that 
the alternative is true, then any less skeptical observer would also be convinced. Therefore, ceasing enrollment and possibly collection 
of additional data in order to assess whether the treatment is beneficial would be a reasonable action from almost any rational perspective.
%
Similarly, if at any point data sufficiently convince an observer whose prior belief is in accordance with $\pi_{E}(\theta)$ that 
the effect of interest is significantly less than what was originally believed, then any less enthusiastic observer would be similarly convinced and ceasing the collection of data altogether would be a reasonable action from almost any rational perspective.

\subsubsection{Maximum Sample Size and Formal Stoppage Criteria}
In this section we formalize stopping criteria for futility and efficacy and give general 
advice for specifying a maximum sample size for the trial.
%
Although sequentially monitored trials in principle require no fixed sample size, in practice due to resource 
constraints it will almost always be the case that a maximum sample size exists. 
%
We recommend that (resources permitting) the maximum sample size, denoted by $n_{\text{max}}$, should be chosen so that there is a 
high probability that the trial generates a compelling demonstration from the perspective of the skeptic when in 
fact $\theta \approx \theta_1$ in a scenario where the data are only examined once when the full set of 
outcomes are ascertained.
%
The rationale behind this strategy is that one would want to ensure the trial's sample size is sufficient so that
there is high probability the data collected will provide a compelling demonstration of treatment benefit to observers 
having relatively extreme skepticism regarding the magnitude of treatment benefit a priori.  


For a sequentially monitored trial, observed data are analyzed as often as is feasible in accordance with 
the cost and/or logistical challenges of assembling the necessary data.
%
For example, if an outcome requires adjudication by a committee of clinical experts, it may not be possible to reanalyze the
data after each new patient's outcome is obtained due to scheduling or other constraints on the adjudication panel.
%
In other scenarios, a patient's outcome may be based on a laboratory parameter's change after a fixed period of time
and the rate limiting factor for sequential monitoring will be how quickly samples can be shipped, processed, and entered
into a database for analysis.  
%
The strategies presented here for sequential monitoring are appropriate regardless of how frequently data can be monitored.
%even if the motivation for sequential monitoring is scenarios where frequent monitoring is feasible.

%\textcolor{red}{Note that I am revising this section significantly. Predominately this is due to imprecise language and what
%seems to be unnecessary introduction of additional notation. For example, $\rmn{eff}(\mathbf{D})$ is the same thing as 
%$P_S(\theta>\theta_0|\mathbf{D})$ which has already been defined so there is no need to relabel it. Also, it is not ``efficacy 
%criteria'', it is a posterior probability -- this is what I mean by an imprecise statement. We need to be careful to avoid these.}

\textcolor{red}{Stopping criteria are based on the posterior probability that the treatment effect is in a particular region (i.e., expression \eqref{eq:postprob}) is convincing, that is,
\begin{equation}\label{eq:convinced}
		P_\pi(\theta\in\Theta_i|\mathbf{D}) > 1-\epsilon,
\end{equation} 
where the subscript $\pi$ indicates the probability is calculated based on a prior or posterior distribution for $\theta$.
}
Stopping criteria for efficacy are defined from the perspective of a skeptical observer. 
	%
The skeptic becomes convinced that a treatment is effective if at some point the observed data suggest there is 
a compelling demonstration that the alternative hypothesis is true. 
	%
Formally, the early stopping criteria are met based on data $\mathbf{D}$ when $P_S(\theta>\theta_0|\mathbf{D})>1-\epsilon$.
	%
Note that the evidence must \textit{exceed} the threshold for what defines it as being compelling.
	%
When the evidence in favor of the alternative surpasses this threshold, it may no longer be necessary to 
enroll patients for the purpose of proving treatment efficacy.


Stopping criteria for futility monitoring are defined from the perspective of the enthusiastic observer. At first thought it may seem appealing to stop the trial when the enthusiast becomes convinced that the
null hypothesis is true, that is, that $P_E(\theta\leq\theta_0|\mathbf{D})>1-\epsilon$. 
%
However, when $\theta=\theta_0$, $P_E(\theta\le\theta_0| \mathbf{D})$ approaches $0.5$ for large sample sizes. 
%
Therefore this potential futility criteria would not be satisfiable unless the observed data were consistent with values of
$\theta$ much less than $\theta_0$.
%
For this reason, we consider a different approach.
%
Recalling that $\theta_1$ represents a plausible, clinically meaningful treatment effect, the early stopping criteria are met based on data $\mathbf{D}$ when $P_E\left(\theta<\theta_1| \mathbf{D}\right)>1-\epsilon$. In this case the trial may be stopped due to there being a compelling demonstration that the treatment effect is much less than hypothesized (i.e., $\theta_1$).
%
\subsection{Specifying Monitoring Priors}\label{sec:mps}
\subsubsection{Default Monitoring Priors}\label{sec:def-mon-priors}
The skeptical and enthusiastic monitoring priors defined in Section \ref{sec:MP} have mode value and tail-probability constraints. 
%
However, these constraints alone do not uniquely determine the priors.
%
There are infinitely many distributions which satisfy these conditions.
%
However, the mode and tail constraints do uniquely determine a pair of normal distributions which might serve as a default set of monitoring priors. 
%
%The density for a  normal distribution $\mathcal{N}(\mu,\sigma)$ is $f(\theta)=\frac{1}{\sqrt 2\pi\sigma} \text{exp}(-\frac{1}{2}(\frac{\theta-\mu}{\sigma})^2)$ where $\mu$ is a location parameter and $\sigma>0$ is the standard deviation. 
%
A default enthusiastic monitoring prior satisfying (i) $\text{argmax}_\theta~\pi_E(\theta)=\theta_1$
and (ii) $P_E(\theta > \theta_0)=1-\epsilon$ is the normal distribution with location $\theta_1$ and standard deviation $\sigma=\frac{\theta_1-\theta_0}{\Phi^{-1}(1-\epsilon)}$, where $\Phi^{-1}$ denotes the quantile function of a standard normal.
%
The specification of $\mu$ and $\sigma$ completely determine the density at all points, including the value of the density at the mode which is $f(\theta_1)=\frac{1}{\sqrt{2\pi}\sigma}$.
%
The skeptical monitoring prior is similarly defined, satisfying (i) $\text{argmax}_\theta~\pi_S(\theta)=\theta_0$ and (ii) $P_S(\theta < \theta_1)=1-\epsilon$.
%Specifying the specification of the mean and variance of a normal distribution completely determine the mode value and tail-probabilities, and is therefore sufficient for defining skeptical and enthusiastic priors (see Section \ref{sec:normal_tail_area}). 

Use of normal distributions for the monitoring priors can be motivated by the Bayesian Central Limit Theorem (CLT) \citep{LeCam2000} which states that, under general conditions, the posterior distribution for $\theta$ approaches normality as the sample size increases, regardless of the initial choice of prior.
%
Therefore, a normally distributed monitoring prior is consistent with belief derived from a sufficiently large dataset with maximum likelihood estimate equal to the mode value required by the prior.

\subsubsection{Generalized Normal Distribution}\label{sec:gen_normal}
Despite the aforementioned justification of normally distributed priors, it may be desirable to construct a monitoring prior with different behavior about the  mode than what is possible when using the normal distribution. 
%
Choosing a flattened distribution is appropriate {when one wishes to reflect more uncertainty regarding the likelihood that 
$\theta$ is near $\theta_1$ (relative to what is permitted by the normal distribution), while maintaining the same residual uncertainty 
that $\theta<\theta_0$}. 
%
Similarly, choosing a concentrated distribution is appropriate {when one wishes to reflect} a higher 
degree of certainty that $\theta$ is near $\theta_1$, while maintaining residual uncertainty that $\theta<\theta_0$. 
%

The family of generalized normal distributions, which contains the normal distribution as a special case, is able to accommodate changes in the density value at the mode while still satisfying the mode value and tail probability constraints. 
%
The density for a generalized normal distribution $\mathcal{GN}(\mu,\alpha,\beta)$ is
\begin{align*}
f(\theta)=\frac{\beta}{2\alpha\Gamma(1/\beta)}\exp\left\{-\left(\frac{|\theta-\mu|}{\alpha}\right)^\beta\right\}
\end{align*} where $\mu$ is a location parameter, $\alpha>0$ is a scale parameter, and $\beta>0$ is a shape parameter \citep{Nadarajah2005}. Fixing the location parameter to be the mode value and changing the shape and scale parameters in conjunction can maintain the tail probability constraint while also changing the density's behavior near the mode. 
%
Recall the density at the mode for a default enthusiastic prior is $f(\theta_1)=\frac{1}{\sqrt{2\pi}\sigma}$. 
%
An enthusiastic monitoring prior in the generalized normal family of distributions can have density at the mode equal to $k\times \frac{1}{\sqrt{2\pi}\sigma}$, with $k<1$ indicating a more flattened distribution and $k>1$ indicating a more peaked distribution at the mode, relative to the default normal distribution. 
%
Appendix B details a procedure for parameterizing these flattened and concentrated monitoring priors. 
%
Flattened and concentrated distributions for different choices of $k$ are shown in Figure \ref{fig:figure1}. 

{Panel B of Figure \ref{fig:figure1} presents a concentrated skeptical prior that satisfies 
the mode value and tail probability constraints given in Section~\ref{sec:MP}, and that has $k=1.5$ times the density value at the mode as compared to the default normal distribution. 
Increasing the density value at the mode translates to a more peaked distribution about the mode as compared to the default normal distribution (shown in Panel A). 
%
Panel D of Figure \ref{fig:figure1} presents a flattened enthusiastic prior that satisfies the mode value and tail probability constraints given in Section~\ref{sec:MP}, and
that has $k=0.67$ times the density value at the mode as compared to the default normal distribution. This translates to a distribution that is significantly more flat about the mode value than the default normal distribution (shown in Panel C). 
%
%Although we have focused on an enthusiastic prior in our discussion here, the same ideas apply to skeptical priors as well.
%
The approach we have proposed results in a unique flattened or concentrated prior.

The use of the scale factor $k$ corresponds to beliefs that reflect useful and nuanced perspectives for clinical trial decision making.
%
A concentrated skeptical prior views $\theta_0$ as being more likely than a data-driven (i.e. normally distributed via Bayesian CLT) perspective while still reflecting residual uncertainty that $\theta>\theta_1$.
%
This is a rational perspective for a monitoring prior since the skeptical viewpoint should give substantial preference to the null value.
%
Similarly, a flattened enthusiastic prior views $\theta_1$ as being less likely than a data-driven perspective while still reflecting residual uncertainty that $\theta<\theta_0$.
%
This is a rational perspective for a monitoring prior since even from an enthusiastic viewpoint, one may wish to reflect increased uncertainty regarding the likelihood of values at and around $\theta_1$.
%
While there is no \textit{correct} choice for the scale factor $k$ for either a skeptical or an enthusiastic prior, the authors' choices
of $1.5$ and $0.67$ are relatively extreme perturbations from that afforded by the normal distribution and will be used henceforth to 
demonstrate the methodology proposed. 
	%
Lastly, we note that the default normal, flattened, and concentrated priors all can be truncated while maintaining the mode and tail 
probability constraints.
	%
This will be necessary when the parameter of interest has bounded support (e.g., $\theta$ is a response probability).
}

%\textcolor{red}{Note that I am changing the notation to what I think will make it more understandable but the graphs need to change too. I also struggle
%with how the definition given forces the distribution to be locally non-informative. It seems like the flattening criteria is what causes that. I recommend
%adding in the criteria $\pi_{NI}(\theta_0) \approx \pi_{NI}(\theta_1)$ \textit{as part of the definition} and revising this section accordingly. Another
%take on it would be to require $\pi_{NI}(\theta)$ be approximately equal for all $\theta \in \left[\theta_0,\theta_1\right]$ as a part of the definition.}
%\textcolor{blue}{
%Lastly, we define a \textit{locally non-informative prior} as a prior that satisfies $\pi_{NI}(\theta_0) = \pi_{NI}(\theta_1)$, and where $\pi_{NI}(\theta)$ is approximately equal for all $\theta \in \left[\theta_0,\theta_1\right]$.
%%
%Let $\delta_{\theta} = 2\left(\theta_1 - \theta_0\right)$ (i.e., twice the difference between $\theta_1$ and $\theta_0$).
%%
%Specifically, we consider the locally non-informative prior, denoted by $\pi_{NI}(\theta)$, that suggests $\theta_{m}=\frac{\theta_0+\theta_1}{2}$ is the most likely value of $\theta$ 
%and that reflects the belief of an observer who is all but convinced that $\theta>\theta_{m} -\delta_{\theta}$ a priori and 
%all but convinced that $\theta < \theta_{m} +\delta_{\theta}$. 
%%
%Formally, this is defined as the prior $\pi_{NI}(\theta)$ satisfying
%\begin{equation}\label{eq:niprior}
%\text{argmax}_\theta~\pi_{NI}(\theta)=\frac{\theta_0+\theta_1}{2}\text{ and } 
%  P_{NI}\left(\theta>\theta_{m}-\delta_{\theta}\right)=P_{NI}\left(\theta< \theta_{m}+\delta_{\theta}\right) =1-\epsilon.
%\end{equation}
%This proposed prior is flattened such that there is $50\%$ more mass in the intervals $\left[\theta_{m}-\delta_{\theta},\theta_0\right]$ 
%and $\left[\theta_1,\theta_{m}+\delta_{\theta}\right]$ as compared to a normal distribution having the same mode value and tail probability constraints. 
%%
%A locally non-informative prior is shown in Panel D of Figure \ref{fig:figure1}, and the technical definition is given in Web Appendix B.


%\item If $\gamma>1$ then the probability that $\theta\in(q,\frac{q+\mu}{2})$ increases and the concentration of $\theta$ around the mode value $\mu$ accordingly decreases, resulting in a flatter distribution around the mode value. 
%\item If $\gamma<1$ then the probability that $\theta\in(q,\frac{q+\mu}{2})$ decreases and the concentration of $\theta$ around the mode value $\mu$ accordingly increases, resulting in a more concentrated distribution around the mode value. 
%\item Flattening or concentrating the distribution around the mode value is useful for specifying monitoring priors that reflect varying opinions about the dispersion of $\theta$. 
%\item An example of parameterizing an enthusiastic prior with a $\mathcal{GN}_p(\tilde{\mu},q,\gamma)$ distribution is demonstrated in Figure \ref{fig:figure1}. 
%\item Choosing a concentrated distribution is appropriate when prior belief reflects a higher degree of certainty that $\theta$ is in a narrow range around $\theta_1$, while maintaining residual uncertainty that $\theta<\theta_0$. 
%\item This decreases the variance of the distribution, and can be seen to reflect a conservative opinion about the values of $\theta$. 
%\item Choosing a flattened distribution is appropriate when $\theta$ is more likely to be in a wide range of values around $\theta_1$, while maintaining the same residual uncertainty that $\theta<\theta_0$. 
%\item This increases the variance of the distribution, and can be seen to reflect a liberal opinion about the values of $\theta$. 

%\item The impact of flattening or concentrating the distribution of a monitoring prior on the operating characteristics of a trial is shown in Section \ref{sec:examples}.
%For example, consider creating skeptical and enthusiastic priors for a response probability on domain $[0,1]$, as $
%\pi_S(\theta)=\mathcal{GN}_{p=1-\epsilon,\Theta=[0,1]}(\tilde{\mu}=\theta_0,q=\theta_1,\gamma=1)$ and $
%\pi_E(\theta)=\mathcal{GN}_{p=\epsilon,\Theta=[0,1]}(\tilde{\mu}=\theta_1,q=\theta_0,\gamma=1)$ respectively.

%The normal and generalized normal family of distributions can be truncated to a restricted domain that reflects the research quantity of interest (e.g. a response probability on [0,1]).
%\subsubsection{Application to higher dimensions}
%Suppose the trial from Section \ref{sec:preliminaries} has added a control arm, and let $\eta_0$ and $\eta_1$ be the response proportions for a control and treatment group respectively. Suppose that the risk difference $\theta=\eta_1-\eta_0$ is the parameter of interest, and let $\eta_0$ be a nuisance parameter. Let $\delta_0$ denote a null risk difference and $\delta_1$ denote a highly efficacious risk difference.

\subsubsection{Incorporating Prior Information in the Monitoring Priors}\label{sec:incorporating}
The monitoring priors are constructed based on the quantities $\theta_0$ and $\theta_1$, as well as the definition of a compelling demonstration. 
%
As described previously, prior information may be directly used in the construction of the enthusiastic prior (e.g., choice of $\theta_1$).
%
It also may be desirable to incorporate prior information into the monitoring process when making a determination of when to stop enrollment 
early for efficacy.
%
To facilitate this, we introduce a procedure for modifying the monitoring process such that, if the enthusiastic prior is congruent with observed data, the degree of skepticism can be adaptively lessened.
	%
We propose incorporating prior information into the monitoring process for efficacy through constructing a mixture prior
from the skeptical and enthusiastic priors using a mixing weight that is constructed from \textcolor{black}{a measure} of compatibility between the
observed data and the \textcolor{black}{enthusiastic prior}. 
	%
We define the \textit{adaptive monitoring prior} \textcolor{black}{for efficacy evaluations} as the mixture distribution	
\begin{equation}\label{eq:inference_prior}
	\pi_{AE}\left(\theta\right)=\omega\cdot\pi_E(\theta)+(1 - \omega)\cdot \pi_S(\theta),
\end{equation}
where $\omega\in(0,1)$ is an adaptively determined mixing weight. %In contrast to adaptive weighting, fixed choices of $\omega$ will be used as comparisons for the performance of the adaptive weight monitoring prior in Section \ref{sec:example2}. 
%
\textcolor{black}{The objective of the proposed approach is to create a mixture prior which favors the enthusiastic prior component in cases where high compatibility is observed between the trial data and the enthusiastic prior, and favors the skeptical prior component if the data observed are incompatible with the enthusiastic prior. This approach is motivated by the rationale that the enthusiastic prior reflects a plausible perspective about the treatment's effect, and one that we assume will typically be informed by data (e.g., from adult trials in the case of a planned pediatric trial).} \textcolor{red}{The adaptive monitoring prior $\pi_{AE}(\theta)$ is used to make determinations regarding treatment efficacy in \eqref{eq:convinced}, as a replacement for the traditional skeptical prior $\pi_S(\theta)$ used for efficacy monitoring.}


The adaptive mixing weight $\omega$ is determined by an assessment of prior-data conflict, proposed by \textcolor{black}{Box} \citep{Box1980}, derived using the prior predictive distribution of 
the data which is defined (in our case) using the \textcolor{black}{enthusiastic prior}.
	%
The prior-predictive distribution (also called the marginal likelihood) for \textcolor{red}{replicated data $\mathbf{D}_{\text{rep}}$ reflects the probability of observing hypothetical data $\mathbf{D}_{\text{rep}}$ given 
the assumed data generating mechanism and prior distribution for $\theta$, and is defined formally as
\begin{equation}\label{eq:pred_dist}
p(\mathbf{D}_{\text{rep}}) =\int p(\mathbf{D}_{\text{rep}}|\theta)\pi(\theta)d\theta.
\end{equation}}
\textit{Box's p-value} is defined as the following:
%\begin{equation}
%p=P(p(y)\leq p(y_{obs}))
%\end{equation}
\begin{equation}\label{eq:box_p}
\psi({\textcolor{red}{\mathbf{D}}})=\int {p(\textcolor{red}{\mathbf{D}_{\text{rep}}})}  1[p(\textcolor{red}{\mathbf{D}_{\text{rep}}})\leq p(\textcolor{red}{\mathbf{D}})] d(\textcolor{red}{\mathbf{D}_{\text{rep}}})
%\sum_{\mathbf{D}}p^{(h)}(\mathbf{D})1[p^{(h)}(\mathbf{D})\leq p^{(h)}(\textcolor{red}{\mathbf{D}})]
\end{equation}
%which in the case of discrete data is equal to $\psi({\textcolor{red}{\mathbf{D}}})=\sum_{\mathbf{D}}p(\mathbf{D})1[p(\mathbf{D})\leq p(\textcolor{red}{\mathbf{D}})]$.
%
where $1[A]$ is an indicator that the event $A$ is true.
%
\textcolor{red}{Note that the expression in \eqref{eq:box_p} can be viewed as an expectation of $1[p(\mathbf{D}_{\text{rep}})\leq p(\mathbf{D})]$ with respect to random $\mathbf{D}_{\text{rep}}$ and is thus equal to the probability of observing a dataset $\mathbf{D}_{\text{rep}}$ as or less likely that $\mathbf{D}$. Thus, large values of $\psi({\mathbf{D}})$ indicate the observed data are very well supported by the distribution in \eqref{eq:pred_dist} whereas small values indicate the observed data are not well supported by the distribution in \eqref{eq:pred_dist} and hence the prior $\pi(\theta)$ that is integrated over to obtain it.}
%Box's p-value can be interpreted as the probability of observing data as or more extreme than $\textcolor{red}{\mathbf{D}}$, given the predictive distribution. 
%
%Small values of $\psi({\textcolor{red}{\mathbf{D}}})$ indicate a lack of compatibility or congruency between the prior and the data. 
%
We propose using the \textcolor{black}{enthusiastic prior} $\pi_E(\theta)$ to compute the quantities in \eqref{eq:pred_dist} and \eqref{eq:box_p} to create \textcolor{black}{a compatibility measurement} $\psi^{(E)}(\textcolor{red}{\mathbf{D}})$ which \textcolor{black}{is} used to determine the mixing weight in \eqref{eq:inference_prior}. Use of Box's p-value as a measure of prior-data conflict has been considered previously \citep{PsiodaXue2020}, but not in the context of sequential monitoring or using a mixture prior framework as we proposed here.

\textcolor{black}{
Define the mixing weight $\omega$ given to the \textit{enthusiastic} prior as
\begin{align}\label{eq:omega}
\omega= (1 - \delta)\cdot\psi^{(E)}(\textcolor{red}{\mathbf{D}})
\end{align}
This mixture weight achieves the goal of favoring the enthusiastic component if the trial data are compatible with that prior, and otherwise assigning a higher weight to the skeptical component. The minimum possible mixing weight $\delta$ assigned to the \textit{skeptical} prior is achieved when $\psi^{(E)}(\textcolor{red}{\mathbf{D}})=1$ and is equal to $\delta$. Choices of $\delta$ in $\{0, 0.05, 0.10, 0.15, 0.20, 0.25\}$ are explored in Sections \ref{sec:example2} and \ref{sec:realdataexample}, and general advice for choosing $\delta$ is given in Section \ref{sec:discussion}.}
%
%If $\psi^{(E)}(\textcolor{red}{\mathbf{D}})>\psi^{(S)}(\textcolor{red}{\mathbf{D}})$, then the observed data are more consistent with the enthusiastic prior, which should be given a greater weight in the mixture. Similarly, if $\psi^{(S)}(\textcolor{red}{\mathbf{D}})>\psi^{(E)}(\textcolor{red}{\mathbf{D}})$ then the skeptical prior should be given a greater mixing weight.
%A conservative choice of mixing weight which gives full weight to the skeptical prior (i.e. $\omega=1$) whenever $\psi^{(S)}(\textcolor{red}{\mathbf{D}})\geq \psi^{(E)}(\textcolor{red}{\mathbf{D}})$ is
%\begin{equation}\label{eq:adaptive_prior}
%\omega = 1 - \text{max}\left(0, \psi^{(E)}(\textcolor{red}{\mathbf{D}})-\psi^{(S)}(\textcolor{red}{\mathbf{D}})\right).
%%\omega=\begin{cases} 
%%      1 & \text{if } \psi^{(S)}(\textcolor{red}{\mathbf{D}})\geq \psi^{(E)}(\textcolor{red}{\mathbf{D}}) \\
%%      1-\left[ \psi^{(E)}(\textcolor{red}{\mathbf{D}})-\psi^{(S)}(\textcolor{red}{\mathbf{D}}) \right] &\text{if } \psi^{(S)}(\textcolor{red}{\mathbf{D}})< \psi^{(E)}(\textcolor{red}{\mathbf{D}})
%%   \end{cases}
%\end{equation}
%%
%As can be seen, for the proposed approach the weight given to the enthusiastic component is the \textit{excess probability} in favor of the enthusiastic prior. \textcolor{blue}{Alternatively, an anti-conservative choice for mixing weight is
%\begin{equation}\label{eq:adaptive_prior_delta}
%\omega=\text{max}\left(\delta, \psi^{(S)}(\textcolor{red}{\mathbf{D}})-\psi^{(E)}(\textcolor{red}{\mathbf{D}})\right).
%\end{equation}
%This liberal choice of mixing weight gives $1 - \delta$ weight to the enthusiastic prior whenever the data are shown to be more compatible with the enthusiastic prior, that is, when $\psi^{(E)}(\textcolor{red}{\mathbf{D}})\geq \psi^{(S)}(\textcolor{red}{\mathbf{D}})$. This formulation of the mixing weight will always include some of the skeptical component in the mixture through the quantity $\delta$, which is typically in the range of $(0, 0.25)$.
%}
%
%Bayesian predictive model checking: model checking statistic and reference predictive distribution.




%

%


%Alternatively, $\omega$ can be informed by the data. Let $\hat{\theta}=\rmn{argmax} \{p(\mathbf{D}|\theta)\}$ be the maximum likelihood estimator of $\theta$ given the data.
%%
%%%\begin{mydef}
%Define
%\begin{equation}\label{eq:dynamic_omega}
%\omega=\omega_{min}+(1-\omega_{min})\frac{\pi_S(\hat{\theta})}{\pi_S(\hat{\theta})+\pi_E(\hat{\theta})}
%\end{equation}
%as a mixing weight that dynamically gives preference to the prior that has the higher density evaluated at $\hat{\theta}$ with a minimum weight $\omega_{min}$ given to the skeptical component with $\omega_{min}\in[0,1]$.
%%%\end{mydef}

\subsubsection{Prior Specification for Nuisance Parameters}\label{sec:cond_marg}
%\textcolor{red}{This section confuses me a good deal. In reference to the figure (and hypotheses to be tested), what are $\eta_0$ and $\eta_1$. These things have not be defined.
%	%
%Second, it seems like one would elicit $\pi(\theta)$ to satisfy tail probability constraints but it is not clear how or why they would do that for $\pi(\eta|\theta)$ since 
%this is a conditional distribution for nuisance parameters. 
%	%
%Third, why would $\pi(\eta|\theta)$ be a default skeptical prior as stated in the figure caption. That does not make sense as $\eta$ has nothing to do with the 
%hypotheses being tested.
%	%
%I think this entire section needs to be rewritten and perhaps simulations redone. It is hard for me to tell based on what is written here.
%}
	%
Often there are additional parameters besides the treatment effect $\theta$ that are not of primary interest (i.e. nuisance parameters).
%
It is necessary to elicit a prior distribution $\pi(\theta,\eta)$ for all unknown quantities.
%
The marginal-conditional factorization of the joint prior $\pi(\theta,\eta)=\pi(\theta)\times\pi(\eta|\theta)$ allows direct elicitation of the marginal prior on the treatment effect and provides the ability to incorporate prior information on the nuisance parameters through their conditional distribution given $\theta$.
%
The prior for $\pi(\theta)$ will be a generalized normal distribution that satisfies the aforementioned mode value and tail probability constraints.
%
We propose to define $\pi(\eta|\theta)$ as a generalized normal distribution, with parameters chosen based on shape of the conditional distribution evaluated at the most likely value of $\pi(\theta)$. For example, if $\pi_S(\theta)$ is a skeptical prior, then the location, shape, and scale parameters for a generalized normal distribution for $\pi(\eta|\theta)$ will be chosen based on $\pi(\eta|\theta=\theta_0)$. The location parameter will be the most likely value of $\eta$ when $\theta=\theta_0$ (e.g. $\text{mode}(\eta|\theta=\theta_0)=\eta_0)$, and shape and scale parameters will be chosen to reflect a reasonable amount of uncertainty regarding $\eta$. 
%

If the parameters $\theta$ and $\eta$ are assumed to be independent, then the joint prior can be factored as $\pi(\theta,\eta)=\pi(\theta)\times\pi(\eta)$ and the priors $\pi(\theta)$ and $\pi(\eta)$ can be elicited separately.
%
In some cases this is not possible. For example, suppose that $\theta$ is the risk difference between response probabilities of a treatment group and the control group, and denote the response probability in the control group by $\eta$.
%
In this case $\theta$ and $\eta$  are linked through constrained support (e.g. $0\leq \theta+\eta\leq 1)$. Such a prior specification is demonstrated in Figure \ref{fig:figure5}, and Section \ref{sec:example2model} uses this representation of the joint prior. Panel A shows the marginal distribution $\pi(\theta)$, Panel B shows the conditional distribution $\pi(\eta|\theta=\theta_0)$, and Panel C shows the joint prior $\pi(\theta,\eta)$. In this example, the conditional distribution $\pi(\eta|\theta)$ will look very similar to the marginal distribution of $\pi(\eta)$ except at the boundaries of the parameter space.



%\subsection{Operating Characteristics}
%\textcolor{red}{All of the concepts in this Section are basic and need not take up space in the main paper. I would cut this entire section. Still, I give some comments
%to help with clarity for the future (still, cut this all). There is some useful information in this section but it can be sprinkled in much more briefly when results are presented
%with notation revised according to the changes I made before. Keep the content of Section 2.3.4. Perhaps that can be a brief subsection of its own. The rest can go keeping on minimal discussion where the associated results are first presented.}

%\subsubsection{Expected sample size and trial duration}
%\textcolor{red}{I do not understand what this paragraph is saying. Neither the expected sample size or expected trial duration are rigorously defined. In the absence of that, 
%there is no point for this subsection.}
%Suppose that the trial has interim analyses based on the number of completed outcomes. Let $\mathbf{D}(n)$ denote the data after $n$ completed outcomes are ascertained. Let $n_{\text{initial}}$ be the first instance where either the efficacy criteria \eqref{eq:efficacy_criteria} or futility criteria \eqref{eq:futility_criteria} are satisfied and enrollment is henceforth terminated, or $n_{\text{max}}$ if the efficacy criteria or futility criteria are not satisfied at any point. Let $n_{\text{final}}$ be the final sample size which includes patients whose outcomes were in progress at the time of enrollment termination. The expected trial duration is determined based the projected time needed for subject enrollment and outcome ascertainment for a number of subjects equal to the expected sample size.

%\subsubsection{Posterior mean and coverage probability}
%Any prior of the form \eqref{eq:inference_prior} and the data likelihood $\mathcal{L}(\theta|\mathbf{D})$ are used to evaluate the posterior distribution for $\theta$ denoted $\pi(\theta|\mathbf{D})$. The posterior mean at the final sample size is $E(\pi(\theta|\mathbf{D}(n_{\text{final}}))$. The coverage probability is the probability of the generating value of $\theta$ to be in an equal-tailed $100(1-\alpha)\%$ credible interval, which is the interval $[\theta^{(\alpha/2)},\theta^{(1-\alpha/2)}]$ based on the quantiles of the posterior distribution $\pi(\theta|\mathbf{D}(n_{\text{final}}))$.

%\subsubsection{Probability of efficacy, futility, and inconclusive findings}
%The probability of stopping early for efficacy and futility are the probability that the efficacy criteria \eqref{eq:efficacy_criteria} or futility criteria \eqref{eq:futility_criteria} are satisfied at an interim analysis with data $\mathbf{D}(n_{\text{initial}})$. The final determination of efficacy and futility refer to the corresponding criteria begin satisfied with data $\mathbf{D}(n_{\text{final}})$. To evaluate the frequentist properties of Bayesian designs, type I error and power are based on $\rmn{eff}(\mathbf{D}(n_{\text{initial}}))$.  Inconclusive findings refers to situations where neither the efficacy or futility criteria are satisfied.

%\subsection{Design Considerations}
%\begin{enumerate}
%\item DC \#1
%\item DC \#2
%\item DC \#3
%\item DC \#4
%\item DC \#5
%\end{enumerate}

\section{Examples}\label{sec:examples}

\subsection{Single-Arm Trial with Binary Endpoint}\label{sec:example1}
\subsubsection{Motivating Example}
We consider the T72 pediatric trial ``A Study of the Safety and Efficacy of Infliximab (REMICADE) in Pediatric Patients With Moderately to Severely Active Ulcerative Colitis" (NCT00336492) \citep{Hyams2012} which was conducted between August 2006 and June 2010.
%
The study population was patients ages 6 through 17 with moderate to severe ulcerative colitis defined as having a baseline Mayo score of 6 or above on a scale of 0-12, where higher scores indicate more severe disease activity.
%
A 5mg/kg dose of infliximab was given to patients at weeks 0, 2, and 6.
%
The primary endpoint was clinical response, corresponding to a 3-point or greater decrease in Mayo score from baseline to week 8. 
%
Patients were enrolled over approximately 33.5 months (approximately 1 patient enrolled per 17 days). 
%
The sample size of 60 patients was chosen so that a frequentist 95\% two-sided confidence interval for the response probability would have a half-width of 0.12 if the true response probability is 0.67.
%
The value 0.67 was the observed proportion of responders among adults with the same disease enrolled in the ACT 1 and ACT 2 trials \citep{Rutgeerts2005} who received the same weight-based dose of 5mg/kg (N = 242).
%
Obtaining a 95\% confidence interval that excluded 0.40 was used as the criterion for classifying the results as clinically significant.
%
Clinical response was observed in 44 of 60 (73.3\%) pediatric patients.

\subsubsection{Model Formulation \& Prior Elicitation}\label{sec:example1model} We use this trial as a motivating example to demonstrate the proposed 
framework for sequential monitoring. The data $\mathbf{D}$ are assumed to be comprised of independent Bernoulli random variables having common response 
probability $\theta$. 
%
As mentioned above, the primary hypotheses evaluated in the trial were $H_0:\theta \le 0.4$ and $H_1: \theta>0.4$.
%
For purposes of monitoring, we took $\theta_1=0.67$ consistent with the ACT 1 and ACT 2 trial data. 
%
The example presented in this section make use of a concentrated skeptical prior and a default enthusiastic prior for monitoring.
%
%\textcolor{red}{Make sure this is moved to supplemental appendix and appropriately referenced. See my recent paper in biometrics for examples for how to write references to the online supplementary materials.}
%
%A comparison of design properties based on various combinations of the monitoring priors is given in Web Appendix D.
%
%In this example, we considered an inference prior defined as the mixture \eqref{eq:3partmix} using the values $\omega_S=\omega_E=0.5$ (i.e., a non-adaptive inference prior), which was used to determine the posterior mean and 95\% credible intervals. 
For sequential monitoring, we consider analyzing the accumulating data after every two patients complete follow-up.

The early stopping criteria, as well as any quantity involving the posterior distribution of $\theta$ requires evaluating integrals of the dimension of $\theta$ (or the dimension of $(\theta,\eta)$ in the case of nuisance parameters).
%
For the cases we consider in this paper, these quantities are 1$-$ or 2$-$dimensional integrals which are evaluated using numerical integration in R \citep{R2017} using the \texttt{pracma} package \citep{Borchers2019}.

\subsubsection{Example Paths}
Figure 3 presents violin plots to illustrate the monitoring process for two hypothetical instances of the trial. 
%
Each instance shows the monitoring priors (left-most set of distributions), the posterior distributions at three selected interim analyses (middle sets), and the posterior distributions from the final analysis (right-most set).
	%
Panel A of Figure \ref{fig:figure2} shows the results of a trial with early stopping for efficacy once 30 outcomes are ascertained, and enrollment is henceforth terminated. 
%
The final data (i.e., the data after ongoing patients are followed-up) in this example path no longer meet the criteria for a compelling demonstration of efficacy.
	%
%A discussion of this type of evidence attenuation is given in Section \ref{sec:evid_decrease}.
%
Panel B of Figure \ref{fig:figure2} shows the results of a trial with early stopping for futility once 30 outcomes are ascertained, and enrollment is henceforth terminated. %The final data matches the last interim analysis since enrolled patients would be transitioned off the IP.


\subsubsection{Choice of Monitoring Priors}

\textcolor{black}{Table 1 illustrates properties of the sequential monitoring procedure with different specifications of skeptical and enthusiastic monitoring priors, using the trial design as described in Section \ref{sec:example1model} with $100,000$ simulated trials per value of $\theta$. When the skeptical prior is concentrated, the probability of the efficacy criteria being satisfied when $\theta=0.4$ is lower than when the skeptical prior is given the default specification. When the enthusiastic prior is flattened, the probability of futility criteria being satisfied when $\theta=0.67$ is greater than when the enthusiastic prior is given the default specification, and expected sample sizes at $\theta=0.4$ and $\theta=0.535$ are lower than when the enthusiastic prior is given the default specification. General advice for choosing the specification of these priors is given in Section \ref{sec:discussion}.}

\subsubsection{Preposterior Analysis of Operating Characteristics}\label{sec:ex1.1}
The operating characteristics presented in this section are estimated using \textcolor{black}{the concentrated skeptical prior and the default enthusiastic prior as shown in Table 1.}
%
As shown in Panel A of Figure \ref{fig:ex1.1}, when the true response probability is $\theta_0=0.4$, the probability of stopping the trial early for efficacy is equal to $0.026$, and at $\theta_1=0.67$ it is $0.953$. 
%
% The expected sample size is less than 60 for each value of $\theta$ considered. 
Interim stoppage for either efficacy or futility occurred in each simulation (i.e., the maximum sample size was such that the trial could not reach the maximum). The expected sample sizes are the lowest when the true response probabilities are $\theta_0$ or $\theta_1$. This is because the trial is more likely to be stopped for futility or efficacy when the data are consistent with the skeptical or enthusiastic priors respectively, which have modes at $\theta_0$ and $\theta_1$. 
%
%At $\theta_0=0.4$, the posterior mean estimates using the mixture prior are slightly greater than the true underlying value, and at $\theta_1=0.67$, the posterior estimates using the mixture prior are slightly less than the true underlying value. 
%%
%This slight bias towards the interval $[\theta_0,\theta_1]$ is because those are the values determined to be the most likely a priori by the mixture prior.
% This is because the mixture prior is a weighted combination of priors that have $\theta_0$ and $\theta_1$ as the most likely values, and the enthusiastic component with mode $\theta_1$ contributes bias towards greater values. 
%
% In a similar way, at $\theta_1=0.67$, the posterior estimates using the mixture prior are slightly less than the true underlying value because the skeptical component with mode $\theta_0$ contributes bias towards smaller values. 
%
%The 95\% credible intervals are shown to have coverage probabilities exceeding their nominal level for all values of $\theta$ considered.
%
%For the generating values of $\theta$ considered, the posterior mean shows bias towards the alternative hypothesis.
%%
%This is because the inference prior is still informative towards to alternative hypothesis.
%%
%The coverage probability of an equal-tailed $95\%$ credible interval is shown to cover the true generating value of $\theta$ with more than $95\%$ probability when $\theta$ is near $\theta_0$ or $\theta_1$, and less than $95\%$ of the time when $\theta$ is an intermediate value (e.g. final coverage probability of $93.1\%$ when $\theta=0.535$.
%%
%This is because the mixture prior places high prior probability near $\theta_0$ and $\theta_1$ and less prior probability at intermediate values.

Given that the early stopping criteria was satisfied as an interim analysis, it is of interest to compare the posterior probability of the alternative once patients in follow-up have completed outcomes using the same skeptical prior.
%
It is of particular interest when the threshold for a compelling demonstration is satisfied for an interim analysis but is no longer satisfied once outcomes from patients in progress are ascertained, as was the case in Panel A of Figure \ref{fig:figure2}.
%
The probability of such an occurrence and the difference between the posterior probabilities evaluated at the different time points is shown in Panel B of Figure \ref{fig:ex1.1}.
%
The probability of these cases occurring is reflected by the percent agreement between interim and final results.
%
Recall that when the generating value of theta is $\theta_0=0.4$, the trial is stopped early for efficacy with probability $0.026$. Among these cases, the posterior probability is still greater than $1-\epsilon$ with probability $0.433$. 
%
This means that the completed outcomes from patients in progress at time enrollment was terminated are likely to meaningfully diminish the evidence in favor of efficacy relative to the threshold for what is viewed as compelling.
%
When the generating value of theta is $\theta_1=0.67$, the trial is stopped early for efficacy with probability $0.952$, and among those cases the posterior probability is  greater than $1-\epsilon$ with probability $0.887$.

The distribution of the posterior probability given the final data for these cases demonstrate that even in the cases where there is evidence decrease, the final posterior probability is still similar to the $1-\epsilon=0.975$ threshold.
%
Consider $\theta_1=0.67$: in the $11.3\%$ of situations where there is evidence decrease below the $1-\epsilon=0.975$ threshold, $90\%$ of these cases have a final posterior probability of approximately $0.93$ or greater. 
%
%\textcolor{red}{These analyses show that even if the skeptical prior was used for the final determination of efficacy rather than an inference prior, the final posterior probability of the alternative would be similar to that which triggered stoppage of enrollment, and the level of similarity increases as the underlying response probability $\theta$ increases.}
%
%\subsubsection{Type 1 Error Rate by the Frequency of Data Monitoring}\label{sec:ex1t1e}
%Figure \ref{fig:ex1t1e} shows the probability of stopping early for efficacy and the posterior probability that the alternative hypothesis is true at the final analysis assuming true response probability of $\theta_0=0.4$. 
%%
%The monitoring frequency is one when an interim analysis is made after every completed outcome (i.e. fully sequential), and is 60 (the maximum sample size) if the only analysis is done at the maximum sample size.
%%
%When there is only a single analysis completed at the maximum sample size, the probability of determining efficacy with the skeptical monitoring prior is $1.3\%$.
%%
%%This is because the determination of efficacy is made with an informative skeptical prior.
%%
%%If the determination of efficacy was made with a non-informative prior, then the probability of the efficacy criteria being satisfied with a single analysis completed at the maximum sample size would be $2.5\%$.
%%
%For more frequent data monitoring, the probability of obtaining a compelling demonstration of efficacy at an interim analysis increases slightly, but never far exceeds the typical nominal level. 
%%
%In many of these cases with early stoppage for efficacy with a true $\theta=\theta_0$, the final posterior probability which includes the total follow-up no longer meets the threshold for a compelling demonstration.
%\begin{figure}
%\begin{center}
%
%    \includegraphics[width=6in]{figure4.png}
%    \caption{Probability of stopping early for efficacy and the posterior probability that the alternative hypothesis is true at the final analysis when $\theta=\theta_0$ (SS; mean sample size, Monitor Freq; monitoring frequency).}
%	\label{fig:ex1t1e}
%
%%  \begin{subfigure}{7in}
%%    \centering\includegraphics[width=6in]{figureS1.png}
%%    \caption{Caption text 2}
%%  \end{subfigure}
% 
%\end{center}
%\end{figure}
%%\item The two sides of the discussion: first is what happens during the trial regarding sequential monitoring, such as \% of time stopping early vs. trial done to completion and expected sample size. Second is the final determination of efficacy or futility and how that relates to Type 1 Error and power. 
%%\item  Remember the best case for sequential monitoring is slow enrollment relative to outcome ascertainment. Slow enrollment means there is a benefit to ending trial early and reach a conclusion faster. Outcome ascertainment needs to be somewhat fast to ensure a good \# of outcomes are generated.


\subsection{Parallel Two-Group Design with Binary Endpoint}\label{sec:example2}
\subsubsection{Motivating Example}\label{sec:example2motivating}
We consider the trial ``The Pediatric Lupus Trial of Belimumab Plus Background Standard Therapy (PLUTO)" (NCT01649765) which was conducted between September 2012 and January 2018 \citep{Brunner2020}.
%
The study population was comprised of patients ages 5 though 17 with active systemic lupus erythematosus (SLE), defined as a baseline SELENA SLEDAI score of 6 or above on a scale of 0-105, where higher scores indicate more severe disease activity.
%
Patients were randomized to monthly dosing of either belimumab 10mg/kg or placebo, while continuing to receive standard of care therapy regardless of assignment.
%
The primary endpoint was a dichotomous variable reflecting a 4-point or greater reduction in SELENA SLEDAI score from baseline to week 52. 
%
The original study design included enrollment of 100 patients, the first 24 patients randomized in a 5:1 allocation ratio (belimumab:placebo) and the remaining 76 patients in a 1:1 ratio, resulting in 58 patients randomized to belimumab and 42 to placebo. 
%
The sample size was based on feasibility constraints rather than power considerations.
%
Data from two studies of belimumab in adults having the same disease resulted in a placebo response probability of $0.39$, and a 10mg/kg response probability of $0.51$. 
%
Using these values for the null and hypothesized response probabilities for the treatment group and assuming a response probability of 0.39 for the control group, a frequentist two-sided hypothesis test with confidence level $95\%$ and $80\%$ power would require 266 patients per group.
%
Ultimately, 93 patients were enrolled over approximately 52.5 months (approximately 1 patient enrolled per 17 days).
%
Clinical response was observed in 28 of 53 (52.8\%) of patients randomized to belimumab and in 17 of 40 (43.6\%) of patients randomized to placebo.

\subsubsection{Model Formulation \& Prior Elicitation}\label{sec:example2model}
We use this trial as a template to demonstrate our framework, in particular the performance of the adaptive monitoring prior defined in Section \ref{sec:incorporating}. The adaptive monitoring prior is necessary since the power analysis in Section \ref{sec:example2motivating} shows the need for many more patients than were available; therefore, a strategy for prospective incorporation of prior information must be implemented for the trial to have a chance of providing a compelling demonstration of efficacy through a pre-specified design.
%
The data $\mathbf{D}$ are assumed to be independent Bernoulli random variables with response probability $\eta_0$ for the placebo group and $\eta_1$ for the treatment group, with $\theta=\eta_1-\eta_0$ denoting the difference in response probabilities. 
%
This trial has a superiority hypothesis of treatment to control with null difference in response probabilities, denoted by $\theta_0=0$.
%
An estimate for the pediatric response probability is denoted by $\eta_0=0.39$ (i.e. the sample proportion of responders from the pooled adult studies), and for purposes of monitoring, a plausible, clinically meaningful difference in response probabilities is $\theta_1=0.12$ (i.e. based on the pooled adult study's treatment response probability of $0.51$).% \citep{Furie2011,Navarra2011}.
%

The skeptical monitoring prior is $\pi_S(\theta,\eta_0)=\pi_S(\theta)\times\pi(\eta_0|\theta)$, where $\pi_S(\theta)$ is a concentrated skeptical prior.
%
The enthusiastic monitoring prior is $\pi_E(\theta,\eta_0)=\pi_E(\theta)\times\pi(\eta_0|\theta)$, where $\pi_E(\theta)$ is a default enthusiastic prior. 
%
The conditional prior for the nuisance parameter $\pi(\eta_0|\theta)$ is specified as a flattened prior around the conditional modal value of $\eta_0=0.39$.
%
The probability of concluding efficacy at an interim analysis is made using the adpative monitoring prior as described in Section \ref{sec:incorporating}.
%
%A 3-part mixture inference prior of the form \eqref{eq:3partmix} as described in Section \ref{sec:inferencepriors} was used to estimate the posterior mean and coverage probabilities for $\theta$.
%
%The locally non-informative prior is $\pi_{NI}(\theta,\eta_0)=\pi_{NI}(\theta)\times\pi(\eta_0|\theta)$.
%
%For the skeptical, enthusiastic, and locally non-informative priors, $\pi(\eta_0|\theta)$ is a flattened prior with mode value $0.39$ and tail probability condition $P(\eta_0>0.59 | \theta)=0.025$.

%, where $\pi_S(\theta)\sim\mathcal{GN}_{p=0.975,\Theta=[-1,1]}(\tilde{\mu}=\theta_0,q=\theta_1,\gamma=0.75)$. The value $\gamma=0.75$ reflects a concentrated distribution which was chosen to reflect a conservative opinion for added Type 1 error control. The enthusiastic monitoring prior is $\pi_E(\theta,\eta_0)=\pi_E(\theta)\times\pi(\eta_0|\theta)$, where $\pi_E(\theta)\sim\mathcal{GN}_{p=0.025,\Theta=[-1,1]}(\tilde{\mu}=\theta_1,q=\theta_0,\gamma=1)$. The value $\gamma=1$ was chosen as the default value. For both the skeptical and enthusiastic monitoring prior,$\pi(\eta_0|\theta)\sim\mathcal{GN}_{p=0.975,H=[max(-\theta,0),min(1,1+\theta)]}(\tilde{\mu}=0.39,q=0.59,\gamma=1.5)$. The value $\gamma=1.5$ was chosen to reflect a liberal opinion about the distribution of $\eta_0$. In general, it is recommended to use flattened priors for nuisance parameters.

A maximum sample size of $n_{\text{max}}=100$ was chosen based on the original trial protocol.
%
A minimum sample size of $n_{min}=50$ was chosen to provide an adequate number of placebo controls to be enrolled given the initial 5:1 allocation to the treatment group.
%
An interim analysis is completed after every two patients have outcomes beginning at $n_{min}$.

\subsubsection{Preposterior Analysis of Operating Characteristics}\label{sec:ex2operatingcharacteristics} 
\textcolor{black}{Figure \ref{fig:ex2varyomega}(A) shows the enthusiastic mixture weights $\omega$ by choice of $\delta$ in \eqref{eq:omega} for all combinations of response difference between the IP and PC groups, when the PC group is fixed at a 38\% response rate (16/42 responses). Observe that the highest mixture weights $\omega$ are observed when the response differences are observed to be around $0.12$, which was the mode value for the enthusiastic prior, and have maximum values of $1-\delta$.}

\textcolor{black}{
The operating characteristics presented in this section are estimated using 2,500 simulated trials per value of $\theta$ using the trial design as described in Section \ref{sec:example2model}. The generating response probability in the placebo group was assumed to be 0.39, and the generating response probability in the treatment group was determined based on risk differences $\theta$ in $\{0, 0.03, 0.06, 0.09, 0.12\}$. Figure \ref{fig:ex2varyomega}(B) shows the probability of stopping early for efficacy and the associated sample sizes when using the adaptive monitoring prior \eqref{eq:inference_prior} with different choices of $\delta$ in \eqref{eq:omega}. When $\delta=0$ or $\delta=0.05$, a conclusion of efficacy is made at an interim analysis $24\%$ and $14\%$ of the time respectively, while this value is $7\%$ or lower when $\delta \geq 0.1$. Reductions in expected sample size are seen with lower choices of $\delta$ and higher generated risk differences. When $\delta=0.1$, a demonstration of efficacy is observed in $53\%$ of simulated trials, with an expected sample size of 90.1. Even though this is a modest reduction from the maximum sample size of 100 for this case, even more favorable reductions are possible when enrollment is comparatively slower and/or when follow-up times are comparatively shorter.}

\textcolor{red}{
\subsection{Comparison to Single Analysis with Non-Informative Prior}\label{sec:non-informative}
We consider comparing the adaptive monitoring prior \eqref{eq:inference_prior} with a standard design that uses a non-informative prior to determine treatment efficacy in \eqref{eq:convinced}. To simplify this illustration, we consider a design without interim analyses. The non-informative prior will be a uniform prior over the joint parameter space of $(\theta,\eta)$. In particular, $\pi_{NI}(\theta)$ is a marginal uniform prior on the risk difference $\theta$, with range $[-1, 1]$.}

\textcolor{red}{
When using $\pi_{NI}(\theta)$ in \eqref{eq:convinced}, the probability of concluding treatment efficacy when $\theta=\theta_0$ is $0.025$ (i.e., the type I error rate), and when $\theta=\theta_1$ it is $0.217$ (i.e., the power). This shows that an analysis with the non-informative prior maintains the nominal type I error rate but, for the sample sizes used in the PLUTO trial, provides very low power. When using $\pi_{AE}(\theta)$ with $\delta=0$, the probability of concluding efficacy is $0.189$ when $\theta=\theta_0$ and is $0.675$ when $\theta=\theta_1$. This shows that the adaptive monitoring prior has a higher than nominal type I error rate, and correspondingly has higher power. If an analysis with the non-informative prior was permitted to have this same higher type I error rate by modifying the evidence threshold in equation \eqref{eq:convinced}, the associated power would be similar, at $0.617$.}
\textcolor{blue}{Consider increasing the sample size to 161 subjects per group and changing the residual uncertainty to $\epsilon=0.05$. When using a non-informative prior, the probability of concluding treatment efficacy when $\theta=\theta_0$ is $0.05$ (i.e., the type I error rate), and when $\theta=\theta_1$ it is $0.70$ (i.e., the power). When using $\pi_{AE}(\theta)$ with $\delta=0$, the probability of concluding efficacy is $0.26$ when $\theta=\theta_0$ and is $0.94$ when $\theta=\theta_1$. If an analysis with the non-informative prior was permitted to have this same higher type I error rate by modifying the evidence threshold in equation \eqref{eq:convinced}, the associated power would identical, at $0.94$. } \textcolor{red}{These analyses show that power is comparable across analyses that use different priors when the procedures are held to the same type I error control constraints.}

\textcolor{red}{Although possible to simply modify the critical value used for hypothesis testing within a traditional design framework, we do not advocate this approach. By doing so, one is implicitly deciding what level of evidence the pediatric data provide on their own, whereas a compelling demonstration of treatment effectiveness after synthesis of all pertinent information still requires additional post hoc evaluation. Instead, the proposed approach focuses on the end result, whether or not there is a compelling demonstration of treatment efficacy once all evidence has been synthesized and provides a clear and rigorous framework for synthesis that balances information borrowing with the need to act sensibly in the presence of prior-data conflict.
}
\section{Real Data Example}\label{sec:realdataexample}
\textcolor{black}{
We consider applying the adaptive monitoring prior of \eqref{eq:inference_prior} to the observed outcomes of the PLUTO trial presented in Section \ref{sec:example2}. %The skeptical and enthusiastic priors are defined in the same way as Section \ref{sec:example2model}, which are centered around response differences of 0 and 0.12 respectively. 
Responses were available for 92 patients (one subject in the placebo group had no outcome available due to a protocol violation). Sequential monitoring after every two completed outcomes was conducted after a minimum sample size of 50 had been reached. Results of this analysis by different choice of $\delta$ are shown in Table 2. When $\delta\leq 0.1$, a conclusion of efficacy is made before the maximum sample size of 92. Figure \ref{fig:2dheatmaps}(A) shows Box's $p$-value at the observed data with $90$ completed outcomes to be $0.965$ which translates directly to the value of $\omega$ in the case that $\delta=0$. Figure \ref{fig:2dheatmaps}(B) shows the efficacy posterior probability of $0.979$ when $\delta=0$ so that $\omega=\psi^{(E)}(\textcolor{red}{\mathbf{D}})$.} 

\textcolor{black}{We note that the final sample size is $\ge 90$ for all choices of $\delta$. Thus, in this application, due to the 52-week period of follow-up for the primary outcome and despite the slow enrollment, the impact of sequential monitoring would not have been substantial in terms of shortening the overall trial or reducing the number of patients enrolled. However, it would have nonetheless provided a mechanism for prospective incorporation of external evidence in a pre-specified manner for the trial. In situations where the time-to-outcome ascertainment is shorter and/or enrollment is slower relative to the time-to-outcome ascertainment, greater reductions in sample size would be expected.}
%\textcolor{blue}{
%When $\delta$ is equal to zero, that is, the weight of the enthusiastic component is allowed to approach 1 in \eqref{eq:inference_prior}, then the efficacy criteria is satisfied at 62 ascertained outcomes. At this point the data were shown to be more compatible with the enthusiastic prior than the skeptical prior, so the skeptical weight was reduced the $\delta=0$ in the mixture. As the value of $\delta$ increases, so does the interim sample size where the efficacy criteria is achieved.}
\section{Discussion}\label{sec:discussion}
In this paper, we present a structured framework for specifying monitoring priors and stoppage criteria for a Bayesian sequentially monitored clinical trial that is based on intuitive justification for the design quantities rather than being motivated by having pre-specified frequentist operating characteristics.
%
%To develop this methodology, we present a precise definition of sufficient level of evidence as it relates to the trial hypotheses and relevant external data, and use the generalized normal distribution for the monitoring priors.
%
%The inputs necessary to parameterize the monitoring priors and define the stoppage criteria are the sufficient level of evidence, the boundary null value for the treatment effect, and a plausible, clinically meaningful value for the treatment effect.
%
%This framework provides a reproducible process that emphasizes interpretability of the trial results as it directly relates to the design choices.
%
%We begin the presentation with fixed choices for the skeptical and monitoring priors, and then extend to adaptive mixtures based on assessments of prior-data conflict.
%
Consequently, the choice of monitoring prior and stoppage criteria are the same regardless of the frequency of data monitoring and the number of patients in progress at enrollment termination, although these factors do impact the operating characteristics of the trial.
%
%Even though frequentist operating characteristics are not an explicit focus of the design, we demonstrate that the Bayesian approaches proposed provide good operating characteristics across a wide range of data monitoring frequencies and enrollment patterns.

\textcolor{black}{Our formulation of the enthusiastic prior enforces that there be residual uncertainty that the null hypothesis is true; it demonstrates strong belief about effectiveness of the treatment yet is still consistent with a degree of equipoise. In the extreme case that interim data are observed to be perfectly consistent with the enthusiastic prior, the residual uncertainty that the null hypothesis is true reflected in the adaptive monitoring prior cannot be less than that reflected in the enthusiastic prior itself. This is a critical feature of the design as it enforces the requirement that observed data must demonstrate some degree of efficacy on their own to justify stopping enrollment early. Without maintaining residual uncertainty as we have done when constructing the enthusiastic prior, it would be possible to conclude benefit in cases where observed data are somewhat consistent with that prior (i.e., $\psi^{(E)}(\textcolor{red}{\mathbf{D}}) > 0$) but also consistent with no benefit (or even harm). This is particularly problematic when the observed data contain little information compared to the source that informs the enthusiastic prior (as is often the case in pediatric settings). Thus, the proposed approach provide a desirable assurance that evidence of efficacy must come, at least in part, from both the prior information and the trial data. A conclusion of treatment efficacy is possible only when there is overwhelming treatment benefit observed in the trial data so as to convince a skeptic on that data's own merit, or, in the more likely scenario, some evidence of benefit from the trial data along with reasonable compatibility with the enthusiastic prior.}


Our results in Section \ref{sec:example2} can be compared to \textcolor{black}{a published} post-hoc Bayesian hierarchical analysis \citep{Brunner2020} which used data from two studies of the use of belimumab in adults. Patients in the pediatric trial had 1.5 times the odds of clinical response with 95\% CI (0.6, 3.5), and a meta-analysis of the two adult studies showed an odds ratio of 1.6 with 95\% CI (1.3, 2.1). The analysis used a mixture prior which was a weighted sum of a skeptical prior centered at null effect with effective sample size equal to two pediatric patients and an informative prior resulting from the meta-analysis. When the weight of the informative component was 0.55 and above, efficacy was concluded based on a 95\% credible interval excluding one. The 0.55 weight of the informative component, interpreted as a 55\% weight on the relevance of the adult information to the pediatric population, was determined to be reasonable by the clinical team. Our method contrasts such a post-hoc analysis with the prospective use of a monitoring prior for efficacy which gives weight to the adult data at interim analyses, although both methods show the necessity of information borrowing. %\textcolor{red}{Our analyses show that for such a trial to have any chance of early stopping, it is necessary to borrow information for the skeptical monitoring prior as frequent data monitoring with a default skeptical prior has limited potential to conclude efficacy (see Figure \ref{fig:ex2varyomega}).} %In fact, the adaptive weight skeptical prior used in this example may be too conservative for this setting, and adaptive methods that include more liberal information borrowing could be used.

Although the examples provided are based on superiority trials with binary endpoints and response probabilities as the parameter of interest, the framework applies to any type of data and parameter of interest.
%
Future work will involve demonstrating the framework in Bayesian clinical trials with survival outcomes, such as large cardiovascular outcomes trials where frequent analysis of data may be useful to reduce excessive sample size requirements.


\section*{Software} \label{s:Software}
A GitHub repository (https://github.com/psioda/Bayesian-Sequential-Monitoring) contains the programs and other resources 
needed to reproduce the analyses presented Examples 1 and 2 of this paper. Software was written using R 4.1.0.


\section*{Acknowledgments}

To be written.

\section*{Declaration of Interest Statement}
To be written.

 \bibliographystyle{agsm} 
 \bibliography{./References}		

\newpage
\section*{Appendices}
\subsection*{A: Bayesian Hypothesis Testing}\label{sec:hypothesis}
Consider the hypotheses $H_0:\theta\in\Theta_{0}$ versus $H_1:\theta\in\Theta_{1}$ where $\Theta_{0}\bigcup \Theta_{1} = \Theta$ and $\Theta_{0} \bigcap \Theta_{1} = \emptyset$.
%
Formal Bayesian hypothesis testing requires the specification of prior probabilities on the hypotheses (e.g. $p(H_i)$ for $i=0,1$)
and prior distributions for $\theta$ specified over the parameter space defined with respect to each of the 
hypotheses (e.g. $\pi(\theta \big| H_i)$ for $i=0,1$). 
%

The posterior probability of hypothesis $H_i$ is given by 
\begin{equation}\label{eq:equation1}
p(H_i|\mathbf{D})=\frac{p(\mathbf{D}|H_i)\cdot p(H_i)}{p(\mathbf{D}|H_0)\cdot p(H_0)+p(\mathbf{D}|H_1)\cdot p(H_1)},
\end{equation}
where $p(\mathbf{D}|H_i) = \int_{\Theta_i} p(\mathbf{D}|\theta)\pi(\theta|H_i)d\theta$ is the marginal likelihood associated with hypothesis $H_i$.
%
In practice, most Bayesian hypothesis testing methods are based on the posterior probability of the \textit{event defining $H_i$}.
%
For this approach, one simply needs to specify a prior $\pi\left(\theta\right)$ representing belief about $\theta$ and compute the posterior distribution.
%
The posterior probability that $\theta\in\Theta_i$ is given by
\begin{equation}\label{eq:equation2}
P(\theta\in\Theta_i|\mathbf{D})
%=\frac{\int_{\Theta_i} p(\mathbf{D}|\theta)\pi (\theta)d\theta}{\int_{\Theta}p(\mathbf{D}|\theta)\pi(\theta) d\theta}
=\frac{\int_{\Theta_i} p(\mathbf{D}|\theta)\pi (\theta|\theta\in\Theta_i)d\theta\cdot P(\theta\in\Theta_i)}
      { \sum_{j=0,1} \int_{\Theta_j}p(\mathbf{D}|\theta)\pi (\theta|\theta\in\Theta_j)d\theta\cdot P(\theta\in\Theta_j) }
\end{equation}
where $P(\theta\in\Theta_i)=\int_{\Theta_i}\pi(\theta)d\theta$. 
%
We can readily see that the $P(\theta\in\Theta_i|\mathbf{D})$ is equal to $p(H_i|\mathbf{D})$ if one takes
$p(H_i) =P(\theta\in\Theta_i)$ and $\pi\left(\theta \big| H_i\right) = \pi\left(\theta\big|\theta \in \Theta_i\right)$ for $i=0,1$.
%
If in fact $\pi\left(\theta\right)$ does represent belief about $\theta$, these choices are perhaps the most intuitive and thus 
we should have no reservation referring to $P(\theta\in\Theta_i|\mathbf{D})$ as the probability that hypothesis $H_i$ is true.
\subsection*{B: Parameterizing Flattened and Concentrated Monitoring Priors}\label{sec:gen_normal_details}

Recall the value of the normal density at the mode is $\frac{1}{\sqrt{2\pi}\sigma}$ and note that the value of a generalized normal density at the mode is $\frac{\beta}{2\alpha\Gamma(1/\beta)}$. These are equivalent when $\beta=2$ and $\alpha=\sqrt{2}\sigma$ (i.e. the normal density is a special case of the generalized normal density at these parameter values). Let $F_{\mu,\alpha,\beta}$ denote the cumulative distribution function of the generalized normal distribution $\mathcal{GN}(\mu,\alpha,\beta)$, which can be expressed as \citep{Griffin2018}
\begin{align*}
P(\theta\leq q|\mu,\alpha,\beta)=\frac{1}{2}+\frac{\text{sign}(q-\mu)}{2}\int_0^{|q-\mu|^\beta}\frac{w^{1/\beta-1}}{\alpha\Gamma(1/\beta)}\text{exp}\left\{-\left(\frac{1}{\alpha}\right)^\beta w\right\} dw.
\end{align*}
%
A flattened or concentrated enthusiastic monitoring prior in the generalized normal family of distributions has density at the mode equal to $k\times \frac{1}{\sqrt{2\pi}\sigma}$. 
%
The parameters for the generalized normal distribution $\mathcal{GN}(\mu,\alpha,\beta)$ are derived as follows: $\mu$ remains equal to the mode value of $\theta_1$ and $\alpha$ and $\beta$ are determined to minimize the function 
\begin{align*}
\bigg(F_{\mu,\alpha,\beta}(\theta_0)-\epsilon\bigg)^2+\left(\frac{\beta}{2\alpha\Gamma(1/\beta)}-k \frac{1}{\sqrt{2\pi}\sigma}\right)^2
\end{align*} with box-constrained optimization \citep{Byrd1995}, where $\sigma=\frac{\theta_1-\theta_0}{\Phi^{-1}(1-\epsilon)}$ is the standard deviation of the default normally distributed enthusiastic monitoring prior. The first term reflects the residual uncertainty that $\theta<\theta_0$, and the second term reflects the density at the mode value. Similarly, the parameters for a flattened or concentrated skeptical monitoring prior are as follows: $\mu$ remains equal to the mode value of $\theta_0$ and $\alpha$ and $\beta$ are determined to minimize the function 
\begin{align*}
\bigg((1-F_{\mu,\alpha,\beta}(\theta_1))-\epsilon\bigg)^2+\left(\frac{\beta}{2\alpha\Gamma(1/\beta)}-k \frac{1}{\sqrt{2\pi}\sigma}\right)^2,
\end{align*} 
where $\sigma=\frac{\theta_0-\theta_1}{\Phi^{-1}(\epsilon)}$.
%
%The parameters for a locally non-informative generalized normal distribution are derived as follows: $\mu$ is equal to $\frac{\theta_0+\theta_1}{2}$ (i.e. the midpoint between $\theta_0$ and $\theta_1$) and $\alpha$ and $\beta$ are determined to minimize the function 
%\begin{align*}
%\bigg(F_{\mu,\alpha,\beta}\left(\frac{3\theta_0-\theta_1}{2}\right)-\epsilon\bigg)^2+\left(\frac{\beta}{2\alpha\Gamma(1/\beta)}-k \frac{1}{\sqrt{2\pi}\sigma}\right)^2,
%\end{align*} where $\sigma=\frac{2(\theta_1-\theta_0)}{\Phi^{-1}(1-\epsilon)}$ (i.e. reflecting residual uncertainty that $\theta<\frac{\theta_0+\theta_1}{2}-2(\theta_1-\theta_0)$) and $k=1.5$. Note that the standard deviation $\sigma$ was chosen to be twice that of the normally distributed skeptical or enthusiastic monitoring priors, so that the locally non-informative prior would have greater dispersion around the mode value.

This parameterizing procedure is applicable to a generalized normal distribution truncated to an interval domain (e.g. when $\theta$ is a response probability with domain $[0,1]$). In this case, the generalized normal distribution truncated to an interval domain $\Theta=(\theta_{min},\theta_{max})$ has density equal to $f(\theta)=c\cdot\exp\left\{-\frac{|\theta-\mu|}{\alpha}^{\beta}\right\}{I(\theta\in \Theta)}$ where $c=\frac{\beta}{2\alpha \Gamma(1/\beta)}({F_{\mu,\alpha,\beta}(\theta_{max})-F_{\mu,\alpha,\beta}(\theta_{min})})^{-1}$.

\textcolor{red}{
\subsection*{C: Step-by-Step Implementation Guide}
Below are step-by-step instructions for implementing this method in the context of a non-inferiority trial.
\begin{enumerate}
\item Identify parameters for the trial design:
\begin{enumerate}
\item Specify null treatment effect $\theta_0$ which is used to define the null and alternative hypotheses $H_0$ and $H_1$.
\item Specify threshold for a compelling demonstration $1-\epsilon$.
\item Specify the plausible, clinically meaningful value for the treatment effect $\theta_1$.
\end{enumerate}
\item Create monitoring priors:
\begin{enumerate}
\item Choose prior shape for the skeptical and enthusiastic monitoring priors $\pi_S(\theta)$ and $\pi_E(\theta)$ (e.g. choose $k$ as described in Section \ref{sec:def-mon-priors}). Our recommendation is to use a concentrated specification (i.e., $k=1.5$)  for $\pi_S(\theta)$ and a default specification (i.e., a normal distribution which has asymptotic justification as belief arriving from a hypothetical dataset) for $\pi_E(\theta)$.
\item Solve for the parameters in the generalized normal distributions $\pi_S(\theta)$ and $\pi_E(\theta)$ as described in Section \ref{sec:def-mon-priors} and Appendix B. These parameters are determined by the quantities provided in Step 1 and Step 2(a). The code for this paper shows how these parameters were computed.
\item Repeat Steps 2(a,b) for nuisance parameters as described in Section \ref{sec:cond_marg}.
\item If the adaptive monitoring prior $\pi_{AE}(\theta)$ is to be used, specify the minimum possible mixing weight $\delta$ assigned to the skeptical prior. Our recommendation is $\delta=0.1$ so that a modest weight of at least 0.1 is always given to the skeptical prior.
\end{enumerate}
\item Sequentially monitor the clinical trial:
\begin{enumerate}
\item Iteratively conduct monitoring according to \eqref{eq:convinced}. If the adaptive monitoring prior is used, then compute the mixture weight $\omega$ using \eqref{eq:omega} at each iteration.
\end{enumerate}
\end{enumerate}
}

\newpage
\section*{Tables}

\begin{table}[htbp]\label{tbl:real-pluto}%
\centering
\caption{Operating characteristics of Example 1 simulations based on REMICADE trial. $\pi_S(\theta)=$ skeptical prior, $\pi_E(\theta)=$ enthusiastic prior, Eff = probability of efficacy criteria satisfied at interim analysis, Fut = probability of futility criteria satsified at interim analysis, E(SS) = expected sample size at interim stoppage, Conc = concentrated prior specification, Def = default prior specification, Flat = flattened prior specification. }%
\begin{tabular*}{500pt}{@{\extracolsep\fill}ccccccccccc@{\extracolsep\fill}}%
\toprule
	&		&	&		$\theta=0.4$&			&		&	$\theta=0.535$	&		&	&		$\theta=0.67$&		\\
	\cline{3-5} \cline{6-8} \cline{9-11}
$\pi_S(\theta)$	&	$\pi_E(\theta)$	&	Eff  &	Fut 	&	E(SS) &	Eff  &	Fut 	&	E(SS) 	&	Eff  &	Fut 	&	E(SS) 	\\
\midrule
Conc	&	Def	&	0.026	&	0.974	&	20.9 	&	0.438	&	0.562	&	30.9	&	0.953	&	0.047	&	23.7	\\
Conc	&	Flat	&	0.025	&	0.975	&	18.4 	&	0.397	&	0.603	&	27.7	&	0.930	&	0.070	&	23.0	\\
Def	&	Flat	&	0.033	&	0.967	&	18.3	&	0.428	&	0.572	&	26.3	&	0.934	&	0.066	&	21.3 	\\
Def	&	Def	&	0.037	&	0.963	&	20.7	&	0.478	&	0.522	&	29.2 	&	0.957	&	0.043	&	21.9	\\
\bottomrule
\end{tabular*}
\end{table}

\newpage
\begin{table}[htbp]\label{tbl:real-pluto}%
\centering
\caption{Summary characteristics of re-analysis of PLUTO trial. I/F = Interim/Final, $\psi^{(E)}(\textcolor{red}{\mathbf{D}})$ = Box's $p$-value using enthusiastic prior, $\omega$ = Enthusiastic mixing weight in adaptive monitoring prior, Efficacy Post Prob = Posterior probability of treatment efficacy.}%
\begin{tabular*}{450pt}{@{\extracolsep\fill}cccccc@{\extracolsep\fill}}%
\toprule
$\delta$	&	Sample Size (I/F)			&	$\psi^{(E)}(\textcolor{red}{\mathbf{D}})$ (I/F)			&	$\omega$ (I/F)			&	Efficacy Post Prob (I/F)			\\
\midrule
0.00	&	62	/	90	&	0.914	/	0.965	&	0.914	/	0.965	&	0.980	/	0.979	\\
0.05	&	64	/	92	&	0.876	/	0.934	&	0.833	/	0.887	&	0.976	/	0.962	\\
0.10	&	76	/	92	&	0.941	/	0.934	&	0.847	/	0.841	&	0.975	/	0.951	\\
0.15	&	92	/	92	&	0.934	/	0.934	&	0.794	/	0.794	&	0.940	/	0.940	\\
0.20	&	92	/	92	&	0.934	/	0.934	&	0.747	/	0.747	&	0.928	/	0.928	\\
0.25	&	92	/	92	&	0.934	/	0.934	&	0.701	/	0.701	&	0.917	/	0.917	\\
\bottomrule
\end{tabular*}
\end{table}

%\setlength{\tabcolsep}{4pt}
%\renewcommand{\arraystretch}{1.0}

%\begin{table}[htbp]
%  \centering
%  \caption{Examples A--D: Hypothetical Stage I Analysis Results}
%    \begin{tabular}{cccc|cc|cc|cc|cc|cc|cc}
%		\hline
%		             &                                       &         & 
%															& \multicolumn{2}{c}{$\delta_0=1.00$}
%													    & \multicolumn{2}{c}{$\delta_0=0.75$}
%															& \multicolumn{2}{c}{$\delta_0=0.62$}
%															& \multicolumn{2}{c}{$\delta_0=0.53$}
%															& \multicolumn{2}{c}{$\delta_0=0.47$}
%															& \multicolumn{2}{c}{$\delta_0=0.00$} \\
%			             &                                       &         & 
%															& \multicolumn{2}{c}{$a_{0}^{\text{H}}\approx 0.0$}
%													    & \multicolumn{2}{c}{$a_{0}^{\text{H}} = 0.01$}
%															& \multicolumn{2}{c}{$a_{0}^{\text{H}} = 0.05$}
%															& \multicolumn{2}{c}{$a_{0}^{\text{H}} = 0.15$}
%															& \multicolumn{2}{c}{$a_{0}^{\text{H}} = 0.25$}
%															& \multicolumn{2}{c}{$a_{0}^{\text{H}} = 1.00$} \\														
%        & $\hat{\pi}_0$;$\hat{\theta}$ & $\omega_0$,$\gamma_1$ &  $n_{\text{o}}$ 
%				& $a_0$;$\gamma_2$ & $\gamma_3$
%				& $a_0$;$\gamma_2$ & $\gamma_3$
%				& $a_0$;$\gamma_2$ & $\gamma_3$
%				& $a_0$;$\gamma_2$ & $\gamma_3$
%				& $a_0$;$\gamma_2$ & $\gamma_3$
%				& $a_0$;$\gamma_2$ & $\gamma_3$ \\ \hline
%\rowcolor{Gray} A  & ~0.18;  & 0.01;  & ~2  & 0.00;  & 0.00  & 0.00;  & 0.00  & 0.00;  & 0.00  & 0.00;  & 0.00  & 0.00;  & 0.00  & 0.01;  & 0.00 \\
%\rowcolor{Gray}  & ~0.10~  & 0.20~  & ~4  & 0.20~  & 0.00  & 0.20~  & 0.00  & 0.20~  & 0.00  & 0.20~  & 0.00  & 0.20~  & 0.00  & 0.16~  & 0.00 \\
%\rowcolor{Gray}  &    &    & ~8  &    & 0.02  &    & 0.02  &    & 0.02  &    & 0.02  &    & 0.02  &    & 0.01 \\
%B  & ~0.35;  & 0.14;  & ~2  & 0.00;  & 0.00  & 0.00;  & 0.00  & 0.00;  & 0.00  & 0.00;  & 0.00  & 0.00;  & 0.00  & 0.14;  & 0.10 \\
%   & -0.02~  & 0.02~  & ~4  & 0.02~  & 0.00  & 0.02~  & 0.00  & 0.02~  & 0.00  & 0.03~  & 0.00  & 0.03~  & 0.00  & 0.31~  & 0.19 \\
%   &    &    & ~8  &    & 0.00  &    & 0.00  &    & 0.00  &    & 0.00  &    & 0.00  &    & 0.22 \\
%\rowcolor{Gray} C  & ~0.41;  & 0.83;  & ~2  & 0.02;  & 0.08  & 0.15;  & 0.39  & 0.29;  & 0.40  & 0.46;  & 0.63  & 0.56;  & 0.77  & 0.83;  & 1.00 \\
%\rowcolor{Gray}  & ~0.04~  & 0.07~  & ~4  & 0.19~  & 0.12  & 0.39~  & 0.38  & 0.46~  & 0.44  & 0.59~  & 0.58  & 0.63~  & 0.74  & 0.88~  & 0.99 \\
%\rowcolor{Gray}  &    &    & ~8  &    & 0.14  &    & 0.34  &    & 0.43  &    & 0.63  &    & 0.70  &    & 0.95 \\
%D  & ~0.41;  & 0.97;  & ~2  & 0.56;  & 0.51  & 0.77;  & 0.85  & 0.84;  & 0.85  & 0.90;  & 0.92  & 0.92;  & 1.00  & 0.97;  & 1.00 \\
%   & ~0.10~  & 0.18~  & ~4  & 0.43~  & 0.49  & 0.60~  & 0.75  & 0.65~  & 0.81  & 0.76~  & 0.89  & 0.80~  & 0.96  & 0.95~  & 1.00 \\
%   &    &    & ~8  &    & 0.36  &    & 0.56  &    & 0.72  &    & 0.82  &    & 0.91  &    & 0.99 \\ \hline
%					\multicolumn{16}{l}{Note: $n_{\text{I}}=50$; $n_{\text{max}}=100$; $\hat{\theta} = \hat{\pi}_1-\hat{\pi}_0$; $n_{\text{o}} =$ Number of patients ongoing per group. }
%    \end{tabular}
%  \label{tab:pluto}
%\end{table}

\newpage
\section*{Figures}
\begin{figure}[htbp]
\begin{center}
\includegraphics[width=3in]{./figures/figure1a.png}
\includegraphics[width=3in]{./figures/figure1b.png}
\includegraphics[width=3in]{./figures/figure1c.png}
\includegraphics[width=3in]{./figures/figure1d.png}
\caption{A, Default skeptical prior. B, Concentrated skeptical prior. C, Default enthusiastic prior. D, Flattened enthusiastic prior.}

\label{fig:figure1}
\end{center}
\end{figure}
\newpage
\begin{figure}[htbp]
\begin{center}
%\includegraphics[width=6in]{figure5c.png}
\includegraphics[width=3in]{./figures/figure5a_NEW.png}
\includegraphics[width=3in]{./figures/figure5b_NEW.png}
\includegraphics[width=6in]{./figures/figure5a.png}
%\includegraphics[width=6in]{figure5d.png}
%\includegraphics[width=6in]{figure5b.png}
%\includegraphics[width=6in]{figure5e.png}
\caption{A, Concentrated skeptical prior $\pi_S(\theta)$ truncated to $[-1,1]$. B, Conditional prior $\pi(\eta|\theta=\theta_0)$. C, Joint prior $\pi(\theta,\eta)=\pi(\theta)\times\pi(\eta|\theta)$ truncated based on the conditions $-1<\theta<1$ and $0<\theta+\eta<1$.}
\label{fig:figure5}
 \end{center}
\end{figure}
%
%\begin{figure}[htbp]
%	\centering\includegraphics[scale=0.80]{./figures/HEATMAP-2STAGE.pdf}  
%	\caption{Heat map for $\omega_0$ values based on the prior predictive distribution for the PLUTO trial data. 
%	         Points beneath the dashed black line indicate data that are consistent with treatment harm.
%					 The values of $\omega_0$ are annotated for select data points (depicted with a square) to illustrate
%					 how quickly the values decay. The value for the observed PLUTO trial data are depicted with a star. Note that 
%					 the number of treated and control group responders are restricted to focus on the sample sizes
%					 where $\omega_0$ is non-negligible.} \label{HEATMAP}
%\end{figure}

\begin{figure}[htbp]
\begin{center}
    \includegraphics[width=6in]{./figures/figure2a.png}
    \includegraphics[width=6in]{./figures/figure2b.png}
    \caption{Example paths for the trial described in Section \ref{sec:example1model}. A, Early stoppage for efficacy. B, Early stoppage for futility.}
	\label{fig:figure2}
\end{center}
\end{figure}

\begin{figure}[htbp]
\begin{center}

    \includegraphics[width=6in]{./figures/figure3a.png}
    \includegraphics[width=6in]{./figures/figure3b.png}
    \caption{A, Sequential design properties. (SS; mean sample size, (I); interim analysis, (F); final analysis). B, Distribution of final posterior probability given interim stoppage and evidence decrease ($\%$ AGR; percent of agreement between final and interim posterior probabilities relative to $1-\epsilon$ threshold).}
	\label{fig:ex1.1}

\end{center}
\end{figure}

\begin{figure}[htbp]
\begin{center}
\includegraphics[width=6in]{./figures/3-part-compatibility-2.png}
   \includegraphics[width=6in]{./figures/figure6.png}
%    \includegraphics[width=6in]{figure6a.png}
    \caption{A, Enthusiastic prior mixing weight $\omega$ associated with skeptical prior weight minimum $\delta$ in \eqref{eq:omega} by observed response difference between IP and PC groups, when the PC response rate is fixed at $38\%$ (16/42 responses). B, Operating characteristics for designs having with skeptical prior weight minimum $\delta$ in \eqref{eq:omega} by true risk difference when the PC response rate generated at $39\%$.}
\label{fig:ex2varyomega}
 \end{center}
\end{figure}



\begin{figure}[htbp]
\begin{center}
   \includegraphics[width=5.3in]{./figures/2dbayesp.png}
    \includegraphics[width=5.3in]{./figures/2dpostp.png}
    \caption{A, Box's p-value by control and treatment sample proportions at the final analysis with 90 subjects when $\delta=0$ is used \eqref{eq:omega} for the adaptive monitoring prior. B, Posterior probability of efficacy by control and treatment sample proportions.}
\label{fig:2dheatmaps}
 \end{center}
\end{figure}

\newpage
\section*{Figure Captions}
Figure 1: {A, Default skeptical prior. B, Concentrated skeptical prior. C, Default enthusiastic prior. D, Flattened enthusiastic prior.}
\\\\
Figure 2: {A, Concentrated skeptical prior $\pi_S(\theta)$ truncated to $[-1,1]$. B, Conditional prior $\pi(\eta|\theta=\theta_0)$. C, Joint prior $\pi(\theta,\eta)=\pi(\theta)\times\pi(\eta|\theta)$ truncated based on the conditions $-1<\theta<1$ and $0<\theta+\eta<1$.}
\\\\
Figure 3: {Example paths for the trial described in Section \ref{sec:example1model}. A, Early stoppage for efficacy. B, Early stoppage for futility.}
\\\\
Figure 4: {A, Sequential design properties. (SS; mean sample size, (I); interim analysis, (F); final analysis). B, Distribution of final posterior probability given interim stoppage and evidence decrease ($\%$ AGR; percent of agreement between final and interim posterior probabilities relative to $1-\epsilon$ threshold).}
\\\\
Figure 5: {A, Enthusiastic prior mixing weight $\omega$ associated with skeptical prior weight minimum $\delta$ in \eqref{eq:omega} by observed response difference between IP and PC groups, when the PC response rate is fixed at $38\%$ (16/42 responses). B, Operating characteristics for designs with skeptical prior weight minimum $\delta$ in \eqref{eq:omega} by true risk difference when the PC response rate generated at $39\%$.}
\\\\
Figure 6: {A, Box's p-value by control and treatment sample proportions at the final analysis with 90 subjects when $\delta=0$ is used \eqref{eq:omega} for the adaptive monitoring prior. B, Posterior probability of efficacy by control and treatment sample proportions.}
\end{document}