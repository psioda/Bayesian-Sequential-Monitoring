\documentclass[12pt]{article}
\usepackage{amsmath,amsfonts,nicefrac}
\usepackage{graphicx}
\usepackage{enumerate}
\usepackage{natbib}
\usepackage{url} % not crucial - just used below for the URL 
\usepackage{ifthen}

%\pdfminorversion=4
% NOTE: To produce blinded version, replace "0" with "1" below.
\newcommand{\blind}{1}
% DON'T change margins - should be 1 inch all around.
\addtolength{\oddsidemargin}{-.5in}%
\addtolength{\evensidemargin}{-.5in}%
\addtolength{\textwidth}{1in}%
\addtolength{\textheight}{-.3in}%
\addtolength{\topmargin}{-.8in}%


\usepackage[table]{xcolor}% http://ctan.org/pkg/xcolor

\newcommand{\cl}[2]{\cellcolor{#1!#2}}
\newcommand{\inc}[2]{ \ifthenelse{\equal{#1}{1}}{\input{./sections/#2}}{ } }



\begin{document}

%\bibliographystyle{natbib}

\def\spacingset#1{\renewcommand{\baselinestretch}%
{#1}\small\normalsize} \spacingset{1}


%%%%%%%%%%%%%%%%%%%%%%%%%%%%%%%%%%%%%%%%%%%%%%%%%%%%%%%%%%%%%%%%%%%%%%%%%%%%%%

\if1\blind
{
  \title{\bf Towards Structured Use of Bayesian Sequential Monitoring in Clinical Trials}
  \author{Evan Kwiatkowski\textsuperscript{$\dagger$}, 
	        Eugenio Andraca-Carrera\textsuperscript{$\ddagger$},\\
					Mat Soukup\textsuperscript{$\ddagger$},
					\medskip Matthew A. Psioda\textsuperscript{$\dagger$}\thanks{The authors gratefully acknowledge \textit{please remember to list all relevant funding sources in the unblinded version}}\\
	  %
	  $\dagger$ Department of Biostatistics,
		University of North Carolina, \\
		McGavran-Greenberg Hall, CB\#7420, \\
		%
		\medskip Chapel Hill, North Carolina, U.S.A.\\
    $\ddagger$ Division of Biometrics VII, Office of Biostatistics \\
		           Center for Drug Evaluation and Research, \\
							 US Food and Drug Administration, \\
							 Silver Spring, Maryland, USA \\									
		}
  \maketitle
} \fi

\if0\blind
{
  \bigskip
  \bigskip
  \bigskip
  \begin{center}
    {\LARGE\bf Title}
\end{center}
  \medskip
} \fi

\bigskip
\begin{abstract}
The text of your abstract. 200 or fewer words.
\end{abstract}

\noindent%
{\it Keywords:}  3 to 6 keywords, that do not appear in the title
\vfill

\newpage
\spacingset{1.5} % DON'T change the spacing!



\section{Introduction}

Things to discuss:
\begin{itemize}
 \item 21\textsuperscript{st} Century Cures Act (MATT)
 \item PDUFA VI reauthorization (MATT)
 \item Expansive work already done on sequential monitoring  (EVAN -- draft on 6/21)
 \item Our majors contribution (EVAN -- as early as possible in introduction without having the flow appear weird -- draft on 6/21)
 \item Outline for the remaining section of the paper (EVAN -- draft on 6/21)
\end{itemize}

\section{Methods}

As you introduce ideas that come from or extend other ideas in the literature, cite the relevant literature.

\subsection{Monitoring versus Estimation Priors (EVAN -- draft on 6/21)}

\begin{itemize}
 \item Define generally in terms of $\boldsymbol\theta = \left( \gamma, \boldsymbol\psi  \right)$ where $\gamma$ is a parameter of interest
       and $\boldsymbol\psi$ is a nuisance parameter (possible vector valued).
 \item Define \textit{Monitoring} Priors and \textit{Inference} Priors.
 \item Make connection between Inference priors and two-part mixture prior and BMA.
 \item Define \textit{Skeptical} and \textit{Enthusiastic} monitoring priors and how each would be used.
 \item I would have a generic graphic to illustrate the types of priors and the mixture.
 %\item Motivate in the context of a simple example (i.e., single parameter binary example).
\end{itemize}

Suppose the parameter space is $\Theta$ and consider testing the hypothesis $H_0:\theta\in\Theta_{H_0}$ versus $H_1:\theta\in\Theta_{H_1}$, where $\Theta=\Theta_{H_0}\cup \Theta_{H_0}$ and $\Theta_{H_0}\cap \Theta_{H_0}=\emptyset$. Let $D$ denote the data collected in the experiment, let $\pi\equiv \pi(\theta)$ denote a prior distribution for $\theta$, and let $p(\theta|D,\pi)$ denote the posterior distribution of $\theta$ given a particular prior distribution.

Before any data is collected, define the skeptical viewpoint as being ``all but convinced" that $\theta\in\Theta_{H_0}$ and similarly the enthuastic viewpoint as being ``all but convinced" that $\theta\in\Theta_{H_1}$. These viewpoints can be made rigorous by associating them with priors $\pi_{Skeptical}$ and $\pi_{Enthuastic}$ such that $P(\theta\in\Theta_0|\pi_{Skeptical})$ and $P(\theta\in\Theta_1|\pi_{Enthuastic})$ are sufficiently close to 1. Although these viewpoints are highly certain, we require that $\pi(\theta)>0$ for all $\theta\in\Theta$.


These viewpoints can be used in monitoring the trial once data is collected. It is reasonable to  The efficacy critiera is $P(\theta\in\Theta_1|D,\pi_{Skeptical})\geq c_1$, and the futility criteria is $P(\theta\in\Theta_0|D,\pi_{Enthuastic})\geq c_2$, where $c_1$ and $c_2$ are the thresholds reflecting strong belief in the skeptical and enthuastic viewpoints respectively.



A standard Bayesian decision rule would reject $H_0$ when $P(\theta\in\Theta_1 | D)\geq 0.95$. 

For example, consider testing the hypothesis $H_0:\theta\leq\theta_0$ versus $H_1:\theta>\theta_0$ where $\theta$ is a treatment effect of interest. A standard Bayesian decision rule would reject $H_0$ when $P(\theta>\theta_0|D)\geq 0.95$.

Let $p(H_0)$ and $p(H_1)$ denote the prior probabilities for $H_0$ and $H_1$, where $p(H_0)+p(H_1)=1$. Let $D$ denote the data collected in the experiment. Let $\pi=\pi(\theta)$ denote a prior distribution for $\theta$ and define $p(D|\pi)=\int_\theta L(\theta|D)\pi(\theta)d\theta$ be the marginal likelihood for the data given the prior $\pi$.

For example, consider testing the hypothesis $H_0:\theta\leq\theta_0$ versus $H_1:\theta>\theta_0$ where $\theta$ is a treatment effect of interest. Suppose an effect $\theta_1>\theta_0$ is thought to be highly clinically relevant and plausible given what limited data is available. Consider a clinical trial with a single analysis and fixed sample size chosen that there is a high probability of proving $H_1$ when $\theta=\theta_1$. A standard Bayesian decision rule would rejectg $H_0$ when $P(\theta>\theta_0|D)\geq 0.95$ which will result in a type one error rate of $0.05$ (approximately) if $\theta=\theta_0$ when the analysis prior is non-informative (a so-called reference or flat prior).

\subsection{Futility Monitoring Using Probability of Success (EVAN -- draft on 6/21)}

\begin{itemize}
 \item Futility monitoring using POS is about stopping early when their is a high likelihood
       of a study being inconclusive at the end of the study.
 \item Since the final analysis uses the \textit{Inference} prior, POS should be based on the
       inference prior.
 \item Develop the framework for POS and show how it is a weighted average POS based on the skeptical
       and enthusiastic priors.
\end{itemize}



\section{Examples -- (EVAN)}

\subsection{Single-Arm Oncology Proof-of-Activity Trial w/ Binary Endpoint}

\subsection{Parallel Two-Group Superiority Trial /w Continuous Binary Endpoint}

\subsection{Three-Arm, Placebo Controlled Non-Inferiority Trial w/ Continuous Endpoint}

\section{Discussion -- (MATT/EVAN)}

%\section{Example of Parallel Two-Group Design}
%
%In this section we consider a sequential monitoring approach in a parallel two-group setting using a binary response endpoint.
%
%We consider the case where the goal is prove superiority of an investigational product (IP) to a control 
%which could be a placebo.
%
%Let $\pi_0$ represent the response rate the control group and $\pi_1$ represent the response probability for the IP group.
%
%Here we wish to elicit pessimistic and enthusiastic priors consistent with the following:
% \begin{enumerate}
%  \item The control group response probability is expected to be approximately $0.20$ and investigators are relatively sure
%		    that the it will be between 0.5 and 0.35.
%				
%	\item The IP group response probability is likely to provide an improvement of approximately 0.20.
%\end{enumerate}
%
%In this setting pessimism or optimism is reflected in the induced prior on the difference in proportions $\pi_1 - \pi_0$.
%
%One easy way to specify such a prior in this case is to elicit a bivariate normal prior for $\pi_0$ and $\pi_1$ truncated to the unit square.  


%
%		\begin{figure}
%      \centering\includegraphics[scale=0.70]{./FIGURES/BINARY-PRIORS.pdf}
%      \caption{Prior for Control Group Response Probability \label{fig:pmp}}
%    \end{figure}
			
	

%\bigskip
\newpage
\begin{center}
{\large\bf SUPPLEMENTARY MATERIAL}
\end{center}


\section{BibTeX}

 \bibliographystyle{agsm}
 \bibliography{./References}		

\end{document}
